\documentclass[12pt, a4paper]{article}

\usepackage[utf8]{inputenc}
\usepackage[T1]{fontenc}
\usepackage[russian]{babel}
\usepackage[oglav,spisok,boldsect,eqwhole,figwhole,hyperref,hyperprint,remarks,greekit]{./style/fn2kursstyle}
\graphicspath{{./style/}{./figures/}{./imgs/}}

\usepackage{multirow}
\usepackage{supertabular}
\usepackage{multicol}

\usepackage{comment}
\usepackage{makecell}
\renewcommand{\arraystretch}{0.9}
\renewcommand\cellset{\renewcommand\arraystretch{0.7}%
\setlength\extrarowheight{2pt}}

\usepackage{float}
\usepackage{filecontents}

% Параметры титульного листа
\author{З.\,И.~Абрамов}
\supervisor{А.\,В.~Чередниченко}
\group{ФН2-52Б}
\date{2022}

% TODO менять в зависимости от лабораторной
\title{5. Квазиньютоновские методы}

\begin{document}

\maketitle

\tableofcontents

\newpage

\section-{Введение}
Найти с заданной точностью точку минимума и минимальное значение целевой функции. Исследовать на минимум квадратичную функцию $10x^2-4xy+7y^2 - - 4\sqrt{5}(5x-y)-16$. Далее исследовать функцию Розенброка $f(x,y)=\alpha (x^2-y)^2+(x\brop{-}1)^2$ с параметрами $\alpha$ 1 и 10. При исследовании для каждой функции брать два параметра точности поиска $\varepsilon=0,01$ и $\varepsilon = 0,000001$. Варианты заданий даны в таблице ниже. Также для каждой функции и каждого параметра точности поиска взять две различные (существенно различные) начальные точки. Начальные точки выбрать самостоятельно.

В результате исследований должно быть выявлено влияние на стоимость методов (количество вычисленных значений целевой функции):
\begin{itemize}
	\item параметров точности поиска;
	\item начальной точки;
	\item выпуклости (переход от квадратичной функции к функции Розенброка);
	\item овражности функции (параметра $\alpha$ в функции Розенброка).
\end{itemize}

Реализовать поиск минимума с помощью методов:
% TODO менять в зависимотсти от лабораторной
Давидона --- Флетчера --- Пауэлла, Бройдена --- Флетчера --- Шенно, Пауэлла.

\newpage

\vspace*{-1cm}

\section{Результаты работы}

\subsection{Квадратичная функция}

\begin{table}[H]
        \centering
        \vspace*{-1.5em}
        \caption{Результаты работы алгоритмов\\для квадратичной функции}
        \footnotesize
        \begin{tabular}{|c|c|c|c|c|}
        \hline
        & &\makecell{Метод\\Ньютона} &\makecell{Модифицир.\\метод Ньютона} &\makecell{Метод\\Марквардта} \\
        \hline
	\multirow{12}{*}{\rotatebox[origin=c]{90}{$\varepsilon = 0.01$}}&\textbf{Начальная точка} &\multicolumn{3}{c|}{\textbf{(-6.00, 2.00)}}\\
	\cline{2-5}
	&Точка минимума &(2.24, 0.00) &(2.24, 0.00) &(2.24, -0.00) \\ 
	\cline{2-5}
	&Минимум &-66.00 &-66.00 &-66.00 \\ 
	\cline{2-5}
	&Кол-во итераций &1 &2 &7 \\ 
	\cline{2-5}
	&\makecell{Кол-во вызовов\\целевой функции} &0 &44 &8 \\ 
	\cline{2-5}
	&\makecell{Кол-во вычислений\\градиента} &2 &3 &8 \\ 
	\cline{2-5}
	&\makecell{Кол-во вычислений\\матриц Гессе} &1 &2 &7 \\ 
	\cline{2-5}
\cline{2-5}&\textbf{Начальная точка} &\multicolumn{3}{c|}{\textbf{(20.00, -30.00)}}\\
	\cline{2-5}
	&Точка минимума &(2.24, 0.00) &(2.24, -0.00) &(2.24, -0.00) \\ 
	\cline{2-5}
	&Минимум &-66.00 &-66.00 &-66.00 \\ 
	\cline{2-5}
	&Кол-во итераций &1 &2 &8 \\ 
	\cline{2-5}
	&\makecell{Кол-во вызовов\\целевой функции} &0 &44 &9 \\ 
	\cline{2-5}
	&\makecell{Кол-во вычислений\\градиента} &2 &3 &9 \\ 
	\cline{2-5}
	&\makecell{Кол-во вычислений\\матриц Гессе} &1 &2 &8 \\ 
	\cline{2-5}
	\hline
	\multirow{12}{*}{\rotatebox[origin=c]{90}{$\varepsilon = 1e-06$}}&\textbf{Начальная точка} &\multicolumn{3}{c|}{\textbf{(-6.000000, 2.000000)}}\\
	\cline{2-5}
	&Точка минимума &(2.236068, 0.000000) &(2.236068, 0.000000) &(2.236068, -0.000000) \\ 
	\cline{2-5}
	&Минимум &-66.000000 &-66.000000 &-66.000000 \\ 
	\cline{2-5}
	&Кол-во итераций &1 &2 &10 \\ 
	\cline{2-5}
	&\makecell{Кол-во вызовов\\целевой функции} &0 &82 &11 \\ 
	\cline{2-5}
	&\makecell{Кол-во вычислений\\градиента} &2 &3 &11 \\ 
	\cline{2-5}
	&\makecell{Кол-во вычислений\\матриц Гессе} &1 &2 &10 \\ 
	\cline{2-5}
\cline{2-5}&\textbf{Начальная точка} &\multicolumn{3}{c|}{\textbf{(20.000000, -30.000000)}}\\
	\cline{2-5}
	&Точка минимума &(2.236068, 0.000000) &(2.236068, -0.000000) &(2.236068, -0.000000) \\ 
	\cline{2-5}
	&Минимум &-66.000000 &-66.000000 &-66.000000 \\ 
	\cline{2-5}
	&Кол-во итераций &1 &2 &10 \\ 
	\cline{2-5}
	&\makecell{Кол-во вызовов\\целевой функции} &0 &82 &11 \\ 
	\cline{2-5}
	&\makecell{Кол-во вычислений\\градиента} &2 &3 &11 \\ 
	\cline{2-5}
	&\makecell{Кол-во вычислений\\матриц Гессе} &1 &2 &10 \\ 
	\cline{2-5}
	\hline

\end{tabular}
\end{table}


            \begin{figure}[H]
	        \centering
	        \includegraphics[width=0.80\textwidth]{Классический метод Ньютона, eps 0.01, start = (-6.00, 2.00), Квадратичная функция}%
	        \caption{Поиск минимума квадратичной функции при $\varepsilon = 0.01$, начальной точке (-6.0, 2.0) классическим методом Ньютона}
	        \vspace*{-1.2cm}
            \end{figure}
            
            \begin{figure}[H]
	        \centering
	        \includegraphics[width=0.80\textwidth]{Модификация метода Ньютона с наискорейшим спуском, eps 0.01, start = (-6.00, 2.00), Квадратичная функция}%
	        \caption{Поиск минимума квадратичной функции при $\varepsilon = 0.01$, начальной точке (-6.0, 2.0) методом Ньютона с наискорейшим спуском}
	        \vspace*{-1.2cm}
            \end{figure}
            
            \begin{figure}[H]
	        \centering
	        \includegraphics[width=0.80\textwidth]{Метод Марквардта, eps 0.01, start = (-6.00, 2.00), Квадратичная функция}%
	        \caption{Поиск минимума квадратичной функции при $\varepsilon = 0.01$, начальной точке (-6.0, 2.0) методом Марквардта}
	        \vspace*{-1.2cm}
            \end{figure}
            
            \begin{figure}[H]
	        \centering
	        \includegraphics[width=0.80\textwidth]{Классический метод Ньютона, eps 0.01, start = (20.00, -30.00), Квадратичная функция}%
	        \caption{Поиск минимума квадратичной функции при $\varepsilon = 0.01$, начальной точке (20.0, -30.0) классическим методом Ньютона}
	        \vspace*{-1.2cm}
            \end{figure}
            
            \begin{figure}[H]
	        \centering
	        \includegraphics[width=0.80\textwidth]{Модификация метода Ньютона с наискорейшим спуском, eps 0.01, start = (20.00, -30.00), Квадратичная функция}%
	        \caption{Поиск минимума квадратичной функции при $\varepsilon = 0.01$, начальной точке (20.0, -30.0) методом Ньютона с наискорейшим спуском}
	        \vspace*{-1.2cm}
            \end{figure}
            
            \begin{figure}[H]
	        \centering
	        \includegraphics[width=0.80\textwidth]{Метод Марквардта, eps 0.01, start = (20.00, -30.00), Квадратичная функция}%
	        \caption{Поиск минимума квадратичной функции при $\varepsilon = 0.01$, начальной точке (20.0, -30.0) методом Марквардта}
	        \vspace*{-1.2cm}
            \end{figure}
            
            \begin{figure}[H]
	        \centering
	        \includegraphics[width=0.80\textwidth]{Классический метод Ньютона, eps 1e-06, start = (-6.000000, 2.000000), Квадратичная функция}%
	        \caption{Поиск минимума квадратичной функции при $\varepsilon = 1e-06$, начальной точке (-6.0, 2.0) классическим методом Ньютона}
	        \vspace*{-1.2cm}
            \end{figure}
            
            \begin{figure}[H]
	        \centering
	        \includegraphics[width=0.80\textwidth]{Модификация метода Ньютона с наискорейшим спуском, eps 1e-06, start = (-6.000000, 2.000000), Квадратичная функция}%
	        \caption{Поиск минимума квадратичной функции при $\varepsilon = 1e-06$, начальной точке (-6.0, 2.0) методом Ньютона с наискорейшим спуском}
	        \vspace*{-1.2cm}
            \end{figure}
            
            \begin{figure}[H]
	        \centering
	        \includegraphics[width=0.80\textwidth]{Метод Марквардта, eps 1e-06, start = (-6.000000, 2.000000), Квадратичная функция}%
	        \caption{Поиск минимума квадратичной функции при $\varepsilon = 1e-06$, начальной точке (-6.0, 2.0) методом Марквардта}
	        \vspace*{-1.2cm}
            \end{figure}
            
            \begin{figure}[H]
	        \centering
	        \includegraphics[width=0.80\textwidth]{Классический метод Ньютона, eps 1e-06, start = (20.000000, -30.000000), Квадратичная функция}%
	        \caption{Поиск минимума квадратичной функции при $\varepsilon = 1e-06$, начальной точке (20.0, -30.0) классическим методом Ньютона}
	        \vspace*{-1.2cm}
            \end{figure}
            
            \begin{figure}[H]
	        \centering
	        \includegraphics[width=0.80\textwidth]{Модификация метода Ньютона с наискорейшим спуском, eps 1e-06, start = (20.000000, -30.000000), Квадратичная функция}%
	        \caption{Поиск минимума квадратичной функции при $\varepsilon = 1e-06$, начальной точке (20.0, -30.0) методом Ньютона с наискорейшим спуском}
	        \vspace*{-1.2cm}
            \end{figure}
            
            \begin{figure}[H]
	        \centering
	        \includegraphics[width=0.80\textwidth]{Метод Марквардта, eps 1e-06, start = (20.000000, -30.000000), Квадратичная функция}%
	        \caption{Поиск минимума квадратичной функции при $\varepsilon = 1e-06$, начальной точке (20.0, -30.0) методом Марквардта}
	        \vspace*{-1.2cm}
            \end{figure}
            \subsection{Функция Розенброка с $\alpha$ = 1}

\begin{table}[H]
        \centering
        \vspace*{-1.5em}
        \caption{Результаты работы алгоритмов\\для функции Розенброка с $\alpha$ = 1}
        \footnotesize
        \begin{tabular}{|c|c|c|c|c|}
        \hline
        & &\makecell{Метод\\Ньютона} &\makecell{Модифицир.\\метод Ньютона} &\makecell{Метод\\Марквардта} \\
        \hline
	\multirow{12}{*}{\rotatebox[origin=c]{90}{$\varepsilon = 0.01$}}&\textbf{Начальная точка} &\multicolumn{3}{c|}{\textbf{(-6.00, 2.00)}}\\
	\cline{2-5}
	&Точка минимума &(1.00, 1.00) &(1.00, 1.00) &(1.00, 1.00) \\ 
	\cline{2-5}
	&Минимум &0.00 &0.00 &0.00 \\ 
	\cline{2-5}
	&Кол-во итераций &5 &7 &12 \\ 
	\cline{2-5}
	&\makecell{Кол-во вызовов\\целевой функции} &0 &151 &13 \\ 
	\cline{2-5}
	&\makecell{Кол-во вычислений\\градиента} &6 &8 &13 \\ 
	\cline{2-5}
	&\makecell{Кол-во вычислений\\матриц Гессе} &5 &7 &12 \\ 
	\cline{2-5}
\cline{2-5}&\textbf{Начальная точка} &\multicolumn{3}{c|}{\textbf{(20.00, -30.00)}}\\
	\cline{2-5}
	&Точка минимума &(1.00, 1.00) &(1.00, 1.00) &(1.00, 1.00) \\ 
	\cline{2-5}
	&Минимум &0.00 &0.00 &0.00 \\ 
	\cline{2-5}
	&Кол-во итераций &4 &11 &11 \\ 
	\cline{2-5}
	&\makecell{Кол-во вызовов\\целевой функции} &0 &233 &12 \\ 
	\cline{2-5}
	&\makecell{Кол-во вычислений\\градиента} &5 &12 &12 \\ 
	\cline{2-5}
	&\makecell{Кол-во вычислений\\матриц Гессе} &4 &11 &11 \\ 
	\cline{2-5}
	\hline
	\multirow{12}{*}{\rotatebox[origin=c]{90}{$\varepsilon = 1e-06$}}&\textbf{Начальная точка} &\multicolumn{3}{c|}{\textbf{(-6.000000, 2.000000)}}\\
	\cline{2-5}
	&Точка минимума &(1.000000, 1.000000) &(1.000000, 1.000000) &(0.999999, 0.999998) \\ 
	\cline{2-5}
	&Минимум &0.000000 &0.000000 &0.000000 \\ 
	\cline{2-5}
	&Кол-во итераций &5 &9 &14 \\ 
	\cline{2-5}
	&\makecell{Кол-во вызовов\\целевой функции} &0 &367 &15 \\ 
	\cline{2-5}
	&\makecell{Кол-во вычислений\\градиента} &6 &10 &15 \\ 
	\cline{2-5}
	&\makecell{Кол-во вычислений\\матриц Гессе} &5 &9 &14 \\ 
	\cline{2-5}
\cline{2-5}&\textbf{Начальная точка} &\multicolumn{3}{c|}{\textbf{(20.000000, -30.000000)}}\\
	\cline{2-5}
	&Точка минимума &(1.000000, 1.000000) &(1.000000, 1.000000) &(1.000000, 1.000000) \\ 
	\cline{2-5}
	&Минимум &0.000000 &0.000000 &0.000000 \\ 
	\cline{2-5}
	&Кол-во итераций &5 &12 &14 \\ 
	\cline{2-5}
	&\makecell{Кол-во вызовов\\целевой функции} &0 &485 &15 \\ 
	\cline{2-5}
	&\makecell{Кол-во вычислений\\градиента} &6 &13 &15 \\ 
	\cline{2-5}
	&\makecell{Кол-во вычислений\\матриц Гессе} &5 &12 &14 \\ 
	\cline{2-5}
	\hline

\end{tabular}
\end{table}


            \begin{figure}[H]
	        \centering
	        \includegraphics[width=0.80\textwidth]{Классический метод Ньютона, eps 0.01, start = (-6.00, 2.00), Функция Розенброка с alpha = 1}%
	        \caption{Поиск минимума функции Розенброка с $\alpha$ = 1 при $\varepsilon = 0.01$, начальной точке (-6.0, 2.0) классическим методом Ньютона}
	        \vspace*{-1.2cm}
            \end{figure}
            
            \begin{figure}[H]
	        \centering
	        \includegraphics[width=0.80\textwidth]{Модификация метода Ньютона с наискорейшим спуском, eps 0.01, start = (-6.00, 2.00), Функция Розенброка с alpha = 1}%
	        \caption{Поиск минимума функции Розенброка с $\alpha$ = 1 при $\varepsilon = 0.01$, начальной точке (-6.0, 2.0) методом Ньютона с наискорейшим спуском}
	        \vspace*{-1.2cm}
            \end{figure}
            
            \begin{figure}[H]
	        \centering
	        \includegraphics[width=0.80\textwidth]{Метод Марквардта, eps 0.01, start = (-6.00, 2.00), Функция Розенброка с alpha = 1}%
	        \caption{Поиск минимума функции Розенброка с $\alpha$ = 1 при $\varepsilon = 0.01$, начальной точке (-6.0, 2.0) методом Марквардта}
	        \vspace*{-1.2cm}
            \end{figure}
            
            \begin{figure}[H]
	        \centering
	        \includegraphics[width=0.80\textwidth]{Классический метод Ньютона, eps 0.01, start = (20.00, -30.00), Функция Розенброка с alpha = 1}%
	        \caption{Поиск минимума функции Розенброка с $\alpha$ = 1 при $\varepsilon = 0.01$, начальной точке (20.0, -30.0) классическим методом Ньютона}
	        \vspace*{-1.2cm}
            \end{figure}
            
            \begin{figure}[H]
	        \centering
	        \includegraphics[width=0.80\textwidth]{Модификация метода Ньютона с наискорейшим спуском, eps 0.01, start = (20.00, -30.00), Функция Розенброка с alpha = 1}%
	        \caption{Поиск минимума функции Розенброка с $\alpha$ = 1 при $\varepsilon = 0.01$, начальной точке (20.0, -30.0) методом Ньютона с наискорейшим спуском}
	        \vspace*{-1.2cm}
            \end{figure}
            
            \begin{figure}[H]
	        \centering
	        \includegraphics[width=0.80\textwidth]{Метод Марквардта, eps 0.01, start = (20.00, -30.00), Функция Розенброка с alpha = 1}%
	        \caption{Поиск минимума функции Розенброка с $\alpha$ = 1 при $\varepsilon = 0.01$, начальной точке (20.0, -30.0) методом Марквардта}
	        \vspace*{-1.2cm}
            \end{figure}
            
            \begin{figure}[H]
	        \centering
	        \includegraphics[width=0.80\textwidth]{Классический метод Ньютона, eps 1e-06, start = (-6.000000, 2.000000), Функция Розенброка с alpha = 1}%
	        \caption{Поиск минимума функции Розенброка с $\alpha$ = 1 при $\varepsilon = 1e-06$, начальной точке (-6.0, 2.0) классическим методом Ньютона}
	        \vspace*{-1.2cm}
            \end{figure}
            
            \begin{figure}[H]
	        \centering
	        \includegraphics[width=0.80\textwidth]{Модификация метода Ньютона с наискорейшим спуском, eps 1e-06, start = (-6.000000, 2.000000), Функция Розенброка с alpha = 1}%
	        \caption{Поиск минимума функции Розенброка с $\alpha$ = 1 при $\varepsilon = 1e-06$, начальной точке (-6.0, 2.0) методом Ньютона с наискорейшим спуском}
	        \vspace*{-1.2cm}
            \end{figure}
            
            \begin{figure}[H]
	        \centering
	        \includegraphics[width=0.80\textwidth]{Метод Марквардта, eps 1e-06, start = (-6.000000, 2.000000), Функция Розенброка с alpha = 1}%
	        \caption{Поиск минимума функции Розенброка с $\alpha$ = 1 при $\varepsilon = 1e-06$, начальной точке (-6.0, 2.0) методом Марквардта}
	        \vspace*{-1.2cm}
            \end{figure}
            
            \begin{figure}[H]
	        \centering
	        \includegraphics[width=0.80\textwidth]{Классический метод Ньютона, eps 1e-06, start = (20.000000, -30.000000), Функция Розенброка с alpha = 1}%
	        \caption{Поиск минимума функции Розенброка с $\alpha$ = 1 при $\varepsilon = 1e-06$, начальной точке (20.0, -30.0) классическим методом Ньютона}
	        \vspace*{-1.2cm}
            \end{figure}
            
            \begin{figure}[H]
	        \centering
	        \includegraphics[width=0.80\textwidth]{Модификация метода Ньютона с наискорейшим спуском, eps 1e-06, start = (20.000000, -30.000000), Функция Розенброка с alpha = 1}%
	        \caption{Поиск минимума функции Розенброка с $\alpha$ = 1 при $\varepsilon = 1e-06$, начальной точке (20.0, -30.0) методом Ньютона с наискорейшим спуском}
	        \vspace*{-1.2cm}
            \end{figure}
            
            \begin{figure}[H]
	        \centering
	        \includegraphics[width=0.80\textwidth]{Метод Марквардта, eps 1e-06, start = (20.000000, -30.000000), Функция Розенброка с alpha = 1}%
	        \caption{Поиск минимума функции Розенброка с $\alpha$ = 1 при $\varepsilon = 1e-06$, начальной точке (20.0, -30.0) методом Марквардта}
	        \vspace*{-1.2cm}
            \end{figure}
            \subsection{Функция Розенброка с $\alpha$ = 10}

\begin{table}[H]
        \centering
        \vspace*{-1.5em}
        \caption{Результаты работы алгоритмов\\для функции Розенброка с $\alpha$ = 10}
        \footnotesize
        \begin{tabular}{|c|c|c|c|c|}
        \hline
        & &\makecell{Метод\\Ньютона} &\makecell{Модифицир.\\метод Ньютона} &\makecell{Метод\\Марквардта} \\
        \hline
	\multirow{12}{*}{\rotatebox[origin=c]{90}{$\varepsilon = 0.01$}}&\textbf{Начальная точка} &\multicolumn{3}{c|}{\textbf{(-6.00, 2.00)}}\\
	\cline{2-5}
	&Точка минимума &(1.00, 1.00) &(1.00, 1.00) &(1.00, 1.00) \\ 
	\cline{2-5}
	&Минимум &0.00 &0.00 &0.00 \\ 
	\cline{2-5}
	&Кол-во итераций &4 &11 &19 \\ 
	\cline{2-5}
	&\makecell{Кол-во вызовов\\целевой функции} &0 &231 &31 \\ 
	\cline{2-5}
	&\makecell{Кол-во вычислений\\градиента} &5 &12 &20 \\ 
	\cline{2-5}
	&\makecell{Кол-во вычислений\\матриц Гессе} &4 &11 &19 \\ 
	\cline{2-5}
\cline{2-5}&\textbf{Начальная точка} &\multicolumn{3}{c|}{\textbf{(20.00, -30.00)}}\\
	\cline{2-5}
	&Точка минимума &(1.00, 1.00) &(1.00, 1.00) &(1.00, 1.01) \\ 
	\cline{2-5}
	&Минимум &0.00 &0.00 &0.00 \\ 
	\cline{2-5}
	&Кол-во итераций &3 &21 &26 \\ 
	\cline{2-5}
	&\makecell{Кол-во вызовов\\целевой функции} &0 &432 &63 \\ 
	\cline{2-5}
	&\makecell{Кол-во вычислений\\градиента} &4 &22 &27 \\ 
	\cline{2-5}
	&\makecell{Кол-во вычислений\\матриц Гессе} &3 &21 &26 \\ 
	\cline{2-5}
	\hline
	\multirow{12}{*}{\rotatebox[origin=c]{90}{$\varepsilon = 1e-06$}}&\textbf{Начальная точка} &\multicolumn{3}{c|}{\textbf{(-6.000000, 2.000000)}}\\
	\cline{2-5}
	&Точка минимума &(1.000000, 1.000000) &(1.000000, 1.000000) &(1.000000, 1.000000) \\ 
	\cline{2-5}
	&Минимум &0.000000 &0.000000 &0.000000 \\ 
	\cline{2-5}
	&Кол-во итераций &5 &12 &20 \\ 
	\cline{2-5}
	&\makecell{Кол-во вызовов\\целевой функции} &0 &483 &32 \\ 
	\cline{2-5}
	&\makecell{Кол-во вычислений\\градиента} &6 &13 &21 \\ 
	\cline{2-5}
	&\makecell{Кол-во вычислений\\матриц Гессе} &5 &12 &20 \\ 
	\cline{2-5}
\cline{2-5}&\textbf{Начальная точка} &\multicolumn{3}{c|}{\textbf{(20.000000, -30.000000)}}\\
	\cline{2-5}
	&Точка минимума &(1.000000, 1.000000) &(1.000000, 1.000000) &(1.000000, 1.000000) \\ 
	\cline{2-5}
	&Минимум &0.000000 &0.000000 &0.000000 \\ 
	\cline{2-5}
	&Кол-во итераций &5 &22 &28 \\ 
	\cline{2-5}
	&\makecell{Кол-во вызовов\\целевой функции} &0 &876 &65 \\ 
	\cline{2-5}
	&\makecell{Кол-во вычислений\\градиента} &6 &23 &29 \\ 
	\cline{2-5}
	&\makecell{Кол-во вычислений\\матриц Гессе} &5 &22 &28 \\ 
	\cline{2-5}
	\hline

\end{tabular}
\end{table}


            \begin{figure}[H]
	        \centering
	        \includegraphics[width=0.80\textwidth]{Классический метод Ньютона, eps 0.01, start = (-6.00, 2.00), Функция Розенброка с alpha = 10}%
	        \caption{Поиск минимума функции Розенброка с $\alpha$ = 10 при $\varepsilon = 0.01$, начальной точке (-6.0, 2.0) классическим методом Ньютона}
	        \vspace*{-1.2cm}
            \end{figure}
            
            \begin{figure}[H]
	        \centering
	        \includegraphics[width=0.80\textwidth]{Модификация метода Ньютона с наискорейшим спуском, eps 0.01, start = (-6.00, 2.00), Функция Розенброка с alpha = 10}%
	        \caption{Поиск минимума функции Розенброка с $\alpha$ = 10 при $\varepsilon = 0.01$, начальной точке (-6.0, 2.0) методом Ньютона с наискорейшим спуском}
	        \vspace*{-1.2cm}
            \end{figure}
            
            \begin{figure}[H]
	        \centering
	        \includegraphics[width=0.80\textwidth]{Метод Марквардта, eps 0.01, start = (-6.00, 2.00), Функция Розенброка с alpha = 10}%
	        \caption{Поиск минимума функции Розенброка с $\alpha$ = 10 при $\varepsilon = 0.01$, начальной точке (-6.0, 2.0) методом Марквардта}
	        \vspace*{-1.2cm}
            \end{figure}
            
            \begin{figure}[H]
	        \centering
	        \includegraphics[width=0.80\textwidth]{Классический метод Ньютона, eps 0.01, start = (20.00, -30.00), Функция Розенброка с alpha = 10}%
	        \caption{Поиск минимума функции Розенброка с $\alpha$ = 10 при $\varepsilon = 0.01$, начальной точке (20.0, -30.0) классическим методом Ньютона}
	        \vspace*{-1.2cm}
            \end{figure}
            
            \begin{figure}[H]
	        \centering
	        \includegraphics[width=0.80\textwidth]{Модификация метода Ньютона с наискорейшим спуском, eps 0.01, start = (20.00, -30.00), Функция Розенброка с alpha = 10}%
	        \caption{Поиск минимума функции Розенброка с $\alpha$ = 10 при $\varepsilon = 0.01$, начальной точке (20.0, -30.0) методом Ньютона с наискорейшим спуском}
	        \vspace*{-1.2cm}
            \end{figure}
            
            \begin{figure}[H]
	        \centering
	        \includegraphics[width=0.80\textwidth]{Метод Марквардта, eps 0.01, start = (20.00, -30.00), Функция Розенброка с alpha = 10}%
	        \caption{Поиск минимума функции Розенброка с $\alpha$ = 10 при $\varepsilon = 0.01$, начальной точке (20.0, -30.0) методом Марквардта}
	        \vspace*{-1.2cm}
            \end{figure}
            
            \begin{figure}[H]
	        \centering
	        \includegraphics[width=0.80\textwidth]{Классический метод Ньютона, eps 1e-06, start = (-6.000000, 2.000000), Функция Розенброка с alpha = 10}%
	        \caption{Поиск минимума функции Розенброка с $\alpha$ = 10 при $\varepsilon = 1e-06$, начальной точке (-6.0, 2.0) классическим методом Ньютона}
	        \vspace*{-1.2cm}
            \end{figure}
            
            \begin{figure}[H]
	        \centering
	        \includegraphics[width=0.80\textwidth]{Модификация метода Ньютона с наискорейшим спуском, eps 1e-06, start = (-6.000000, 2.000000), Функция Розенброка с alpha = 10}%
	        \caption{Поиск минимума функции Розенброка с $\alpha$ = 10 при $\varepsilon = 1e-06$, начальной точке (-6.0, 2.0) методом Ньютона с наискорейшим спуском}
	        \vspace*{-1.2cm}
            \end{figure}
            
            \begin{figure}[H]
	        \centering
	        \includegraphics[width=0.80\textwidth]{Метод Марквардта, eps 1e-06, start = (-6.000000, 2.000000), Функция Розенброка с alpha = 10}%
	        \caption{Поиск минимума функции Розенброка с $\alpha$ = 10 при $\varepsilon = 1e-06$, начальной точке (-6.0, 2.0) методом Марквардта}
	        \vspace*{-1.2cm}
            \end{figure}
            
            \begin{figure}[H]
	        \centering
	        \includegraphics[width=0.80\textwidth]{Классический метод Ньютона, eps 1e-06, start = (20.000000, -30.000000), Функция Розенброка с alpha = 10}%
	        \caption{Поиск минимума функции Розенброка с $\alpha$ = 10 при $\varepsilon = 1e-06$, начальной точке (20.0, -30.0) классическим методом Ньютона}
	        \vspace*{-1.2cm}
            \end{figure}
            
            \begin{figure}[H]
	        \centering
	        \includegraphics[width=0.80\textwidth]{Модификация метода Ньютона с наискорейшим спуском, eps 1e-06, start = (20.000000, -30.000000), Функция Розенброка с alpha = 10}%
	        \caption{Поиск минимума функции Розенброка с $\alpha$ = 10 при $\varepsilon = 1e-06$, начальной точке (20.0, -30.0) методом Ньютона с наискорейшим спуском}
	        \vspace*{-1.2cm}
            \end{figure}
            
            \begin{figure}[H]
	        \centering
	        \includegraphics[width=0.80\textwidth]{Метод Марквардта, eps 1e-06, start = (20.000000, -30.000000), Функция Розенброка с alpha = 10}%
	        \caption{Поиск минимума функции Розенброка с $\alpha$ = 10 при $\varepsilon = 1e-06$, начальной точке (20.0, -30.0) методом Марквардта}
	        \vspace*{-1.2cm}
            \end{figure}
            

\vspace*{1cm}

\section-{Вывод}
% TODO менять в зависимости от лабораторной

\begin{thebibliography}{10}
\bibitem{Attetkov} Аттетков А.В., Галкин С.В., Зарубин С.В. Методы оптимизации. М.: Изд-во МГТУ им Н.Э. Баумана, 2003. 440 с.
\end{thebibliography}

\end{document} 