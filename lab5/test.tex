\subsubsection{Квадратичная функция}

\begin{table}[H]
        \centering
        \vspace*{-1.5em}
        \caption{Результаты работы алгоритмов\\для квадратичной функции}
        \footnotesize
        \begin{tabular}{|c|c|c|c|c|}
        \hline
        & &\makecell{Метод ДФП} &\makecell{Метод БФШ} &\makecell{Метод\\Пауэлла} \\
        \hline
	\multirow{10}{*}{\rotatebox[origin=c]{90}{$\varepsilon = 0.01$}}&\textbf{Начальная точка} &\multicolumn{3}{c|}{\textbf{(-6.00, 2.00)}}\\
	\cline{2-5}
	&Точка минимума &(2.24, -0.00) &(2.24, -0.00) &(2.24, -0.00) \\ 
	\cline{2-5}
	&Минимум &-66.00 &-66.00 &-66.00 \\ 
	\cline{2-5}
	&Кол-во итераций &2 &2 &2 \\ 
	\cline{2-5}
	&\makecell{Кол-во вызовов\\целевой функции} &78 &78 &78 \\ 
	\cline{2-5}
	&\makecell{Кол-во вычислений\\градиента} &3 &3 &3 \\ 
	\cline{2-5}
\cline{2-5}&\textbf{Начальная точка} &\multicolumn{3}{c|}{\textbf{(20.00, -30.00)}}\\
	\cline{2-5}
	&Точка минимума &(2.24, -0.00) &(2.24, -0.00) &(2.24, -0.00) \\ 
	\cline{2-5}
	&Минимум &-66.00 &-66.00 &-66.00 \\ 
	\cline{2-5}
	&Кол-во итераций &2 &2 &2 \\ 
	\cline{2-5}
	&\makecell{Кол-во вызовов\\целевой функции} &77 &77 &77 \\ 
	\cline{2-5}
	&\makecell{Кол-во вычислений\\градиента} &3 &3 &3 \\ 
	\cline{2-5}
	\hline
	\multirow{10}{*}{\rotatebox[origin=c]{90}{$\varepsilon = 1e-06$}}&\textbf{Начальная точка} &\multicolumn{3}{c|}{\textbf{(-6.000000, 2.000000)}}\\
	\cline{2-5}
	&Точка минимума &(2.236068, 0.000000) &(2.236068, 0.000000) &(2.236068, 0.000000) \\ 
	\cline{2-5}
	&Минимум &-66.000000 &-66.000000 &-66.000000 \\ 
	\cline{2-5}
	&Кол-во итераций &2 &2 &2 \\ 
	\cline{2-5}
	&\makecell{Кол-во вызовов\\целевой функции} &116 &116 &116 \\ 
	\cline{2-5}
	&\makecell{Кол-во вычислений\\градиента} &3 &3 &3 \\ 
	\cline{2-5}
\cline{2-5}&\textbf{Начальная точка} &\multicolumn{3}{c|}{\textbf{(20.000000, -30.000000)}}\\
	\cline{2-5}
	&Точка минимума &(2.236068, -0.000000) &(2.236068, 0.000000) &(2.236068, -0.000000) \\ 
	\cline{2-5}
	&Минимум &-66.000000 &-66.000000 &-66.000000 \\ 
	\cline{2-5}
	&Кол-во итераций &3 &3 &3 \\ 
	\cline{2-5}
	&\makecell{Кол-во вызовов\\целевой функции} &170 &171 &170 \\ 
	\cline{2-5}
	&\makecell{Кол-во вычислений\\градиента} &4 &4 &4 \\ 
	\cline{2-5}
	\hline

\end{tabular}
\end{table}


            \begin{figure}[H]
	        \centering
	        \includegraphics[width=0.85\textwidth]{Метод Давидона-Флетчера-Пауэлла, eps 0.01, start = (-6.00, 2.00), Квадратичная функция}%
	        \caption{Поиск минимума квадратичной функции при $\varepsilon = 0.01$, начальной точке (-6.0, 2.0) методом ДФП}
	        \vspace*{-1.2cm}
            \end{figure}
            
            \begin{figure}[H]
	        \centering
	        \includegraphics[width=0.85\textwidth]{Метод Бройдена-Флетчера-Шенно, eps 0.01, start = (-6.00, 2.00), Квадратичная функция}%
	        \caption{Поиск минимума квадратичной функции при $\varepsilon = 0.01$, начальной точке (-6.0, 2.0) методом БФШ}
	        \vspace*{-1.2cm}
            \end{figure}
            
            \begin{figure}[H]
	        \centering
	        \includegraphics[width=0.85\textwidth]{Метод Пауэлла, eps 0.01, start = (-6.00, 2.00), Квадратичная функция}%
	        \caption{Поиск минимума квадратичной функции при $\varepsilon = 0.01$, начальной точке (-6.0, 2.0) методом Пауэлла}
	        \vspace*{-1.2cm}
            \end{figure}
            
            \begin{figure}[H]
	        \centering
	        \includegraphics[width=0.85\textwidth]{Метод Давидона-Флетчера-Пауэлла, eps 0.01, start = (20.00, -30.00), Квадратичная функция}%
	        \caption{Поиск минимума квадратичной функции при $\varepsilon = 0.01$, начальной точке (20.0, -30.0) методом ДФП}
	        \vspace*{-1.2cm}
            \end{figure}
            
            \begin{figure}[H]
	        \centering
	        \includegraphics[width=0.85\textwidth]{Метод Бройдена-Флетчера-Шенно, eps 0.01, start = (20.00, -30.00), Квадратичная функция}%
	        \caption{Поиск минимума квадратичной функции при $\varepsilon = 0.01$, начальной точке (20.0, -30.0) методом БФШ}
	        \vspace*{-1.2cm}
            \end{figure}
            
            \begin{figure}[H]
	        \centering
	        \includegraphics[width=0.85\textwidth]{Метод Пауэлла, eps 0.01, start = (20.00, -30.00), Квадратичная функция}%
	        \caption{Поиск минимума квадратичной функции при $\varepsilon = 0.01$, начальной точке (20.0, -30.0) методом Пауэлла}
	        \vspace*{-1.2cm}
            \end{figure}
            
            \begin{figure}[H]
	        \centering
	        \includegraphics[width=0.85\textwidth]{Метод Давидона-Флетчера-Пауэлла, eps 1e-06, start = (-6.000000, 2.000000), Квадратичная функция}%
	        \caption{Поиск минимума квадратичной функции при $\varepsilon = 1e-06$, начальной точке (-6.0, 2.0) методом ДФП}
	        \vspace*{-1.2cm}
            \end{figure}
            
            \begin{figure}[H]
	        \centering
	        \includegraphics[width=0.85\textwidth]{Метод Бройдена-Флетчера-Шенно, eps 1e-06, start = (-6.000000, 2.000000), Квадратичная функция}%
	        \caption{Поиск минимума квадратичной функции при $\varepsilon = 1e-06$, начальной точке (-6.0, 2.0) методом БФШ}
	        \vspace*{-1.2cm}
            \end{figure}
            
            \begin{figure}[H]
	        \centering
	        \includegraphics[width=0.85\textwidth]{Метод Пауэлла, eps 1e-06, start = (-6.000000, 2.000000), Квадратичная функция}%
	        \caption{Поиск минимума квадратичной функции при $\varepsilon = 1e-06$, начальной точке (-6.0, 2.0) методом Пауэлла}
	        \vspace*{-1.2cm}
            \end{figure}
            
            \begin{figure}[H]
	        \centering
	        \includegraphics[width=0.85\textwidth]{Метод Давидона-Флетчера-Пауэлла, eps 1e-06, start = (20.000000, -30.000000), Квадратичная функция}%
	        \caption{Поиск минимума квадратичной функции при $\varepsilon = 1e-06$, начальной точке (20.0, -30.0) методом ДФП}
	        \vspace*{-1.2cm}
            \end{figure}
            
            \begin{figure}[H]
	        \centering
	        \includegraphics[width=0.85\textwidth]{Метод Бройдена-Флетчера-Шенно, eps 1e-06, start = (20.000000, -30.000000), Квадратичная функция}%
	        \caption{Поиск минимума квадратичной функции при $\varepsilon = 1e-06$, начальной точке (20.0, -30.0) методом БФШ}
	        \vspace*{-1.2cm}
            \end{figure}
            
            \begin{figure}[H]
	        \centering
	        \includegraphics[width=0.85\textwidth]{Метод Пауэлла, eps 1e-06, start = (20.000000, -30.000000), Квадратичная функция}%
	        \caption{Поиск минимума квадратичной функции при $\varepsilon = 1e-06$, начальной точке (20.0, -30.0) методом Пауэлла}
	        \vspace*{-1.2cm}
            \end{figure}
            \subsubsection{Функция Розенброка с $\alpha$ = 1}

\begin{table}[H]
        \centering
        \vspace*{-1.5em}
        \caption{Результаты работы алгоритмов\\для функции Розенброка с $\alpha$ = 1}
        \footnotesize
        \begin{tabular}{|c|c|c|c|c|}
        \hline
        & &\makecell{Метод ДФП} &\makecell{Метод БФШ} &\makecell{Метод\\Пауэлла} \\
        \hline
	\multirow{10}{*}{\rotatebox[origin=c]{90}{$\varepsilon = 0.01$}}&\textbf{Начальная точка} &\multicolumn{3}{c|}{\textbf{(-6.00, 2.00)}}\\
	\cline{2-5}
	&Точка минимума &(1.01, 1.01) &(1.01, 1.01) &(1.01, 1.01) \\ 
	\cline{2-5}
	&Минимум &0.00 &0.00 &0.00 \\ 
	\cline{2-5}
	&Кол-во итераций &12 &12 &12 \\ 
	\cline{2-5}
	&\makecell{Кол-во вызовов\\целевой функции} &436 &436 &436 \\ 
	\cline{2-5}
	&\makecell{Кол-во вычислений\\градиента} &13 &13 &13 \\ 
	\cline{2-5}
\cline{2-5}&\textbf{Начальная точка} &\multicolumn{3}{c|}{\textbf{(20.00, -30.00)}}\\
	\cline{2-5}
	&Точка минимума &(1.00, 0.99) &(0.99, 0.98) &(1.00, 0.99) \\ 
	\cline{2-5}
	&Минимум &0.00 &0.00 &0.00 \\ 
	\cline{2-5}
	&Кол-во итераций &15 &9 &16 \\ 
	\cline{2-5}
	&\makecell{Кол-во вызовов\\целевой функции} &541 &332 &576 \\ 
	\cline{2-5}
	&\makecell{Кол-во вычислений\\градиента} &16 &10 &17 \\ 
	\cline{2-5}
	\hline
	\multirow{10}{*}{\rotatebox[origin=c]{90}{$\varepsilon = 1e-06$}}&\textbf{Начальная точка} &\multicolumn{3}{c|}{\textbf{(-6.000000, 2.000000)}}\\
	\cline{2-5}
	&Точка минимума &(1.000001, 1.000002) &(1.000000, 1.000000) &(1.000001, 1.000001) \\ 
	\cline{2-5}
	&Минимум &0.000000 &0.000000 &0.000000 \\ 
	\cline{2-5}
	&Кол-во итераций &33 &25 &35 \\ 
	\cline{2-5}
	&\makecell{Кол-во вызовов\\целевой функции} &1807 &1373 &1915 \\ 
	\cline{2-5}
	&\makecell{Кол-во вычислений\\градиента} &34 &26 &36 \\ 
	\cline{2-5}
\cline{2-5}&\textbf{Начальная точка} &\multicolumn{3}{c|}{\textbf{(20.000000, -30.000000)}}\\
	\cline{2-5}
	&Точка минимума &(0.999999, 0.999998) &(0.999999, 0.999998) &(0.999999, 0.999998) \\ 
	\cline{2-5}
	&Минимум &0.000000 &0.000000 &0.000000 \\ 
	\cline{2-5}
	&Кол-во итераций &35 &25 &36 \\ 
	\cline{2-5}
	&\makecell{Кол-во вызовов\\целевой функции} &1916 &1376 &1970 \\ 
	\cline{2-5}
	&\makecell{Кол-во вычислений\\градиента} &36 &26 &37 \\ 
	\cline{2-5}
	\hline

\end{tabular}
\end{table}


            \begin{figure}[H]
	        \centering
	        \includegraphics[width=0.85\textwidth]{Метод Давидона-Флетчера-Пауэлла, eps 0.01, start = (-6.00, 2.00), Функция Розенброка с alpha = 1}%
	        \caption{Поиск минимума функции Розенброка с $\alpha$ = 1 при $\varepsilon = 0.01$, начальной точке (-6.0, 2.0) методом ДФП}
	        \vspace*{-1.2cm}
            \end{figure}
            
            \begin{figure}[H]
	        \centering
	        \includegraphics[width=0.85\textwidth]{Метод Бройдена-Флетчера-Шенно, eps 0.01, start = (-6.00, 2.00), Функция Розенброка с alpha = 1}%
	        \caption{Поиск минимума функции Розенброка с $\alpha$ = 1 при $\varepsilon = 0.01$, начальной точке (-6.0, 2.0) методом БФШ}
	        \vspace*{-1.2cm}
            \end{figure}
            
            \begin{figure}[H]
	        \centering
	        \includegraphics[width=0.85\textwidth]{Метод Пауэлла, eps 0.01, start = (-6.00, 2.00), Функция Розенброка с alpha = 1}%
	        \caption{Поиск минимума функции Розенброка с $\alpha$ = 1 при $\varepsilon = 0.01$, начальной точке (-6.0, 2.0) методом Пауэлла}
	        \vspace*{-1.2cm}
            \end{figure}
            
            \begin{figure}[H]
	        \centering
	        \includegraphics[width=0.85\textwidth]{Метод Давидона-Флетчера-Пауэлла, eps 0.01, start = (20.00, -30.00), Функция Розенброка с alpha = 1}%
	        \caption{Поиск минимума функции Розенброка с $\alpha$ = 1 при $\varepsilon = 0.01$, начальной точке (20.0, -30.0) методом ДФП}
	        \vspace*{-1.2cm}
            \end{figure}
            
            \begin{figure}[H]
	        \centering
	        \includegraphics[width=0.85\textwidth]{Метод Бройдена-Флетчера-Шенно, eps 0.01, start = (20.00, -30.00), Функция Розенброка с alpha = 1}%
	        \caption{Поиск минимума функции Розенброка с $\alpha$ = 1 при $\varepsilon = 0.01$, начальной точке (20.0, -30.0) методом БФШ}
	        \vspace*{-1.2cm}
            \end{figure}
            
            \begin{figure}[H]
	        \centering
	        \includegraphics[width=0.85\textwidth]{Метод Пауэлла, eps 0.01, start = (20.00, -30.00), Функция Розенброка с alpha = 1}%
	        \caption{Поиск минимума функции Розенброка с $\alpha$ = 1 при $\varepsilon = 0.01$, начальной точке (20.0, -30.0) методом Пауэлла}
	        \vspace*{-1.2cm}
            \end{figure}
            
            \begin{figure}[H]
	        \centering
	        \includegraphics[width=0.85\textwidth]{Метод Давидона-Флетчера-Пауэлла, eps 1e-06, start = (-6.000000, 2.000000), Функция Розенброка с alpha = 1}%
	        \caption{Поиск минимума функции Розенброка с $\alpha$ = 1 при $\varepsilon = 1e-06$, начальной точке (-6.0, 2.0) методом ДФП}
	        \vspace*{-1.2cm}
            \end{figure}
            
            \begin{figure}[H]
	        \centering
	        \includegraphics[width=0.85\textwidth]{Метод Бройдена-Флетчера-Шенно, eps 1e-06, start = (-6.000000, 2.000000), Функция Розенброка с alpha = 1}%
	        \caption{Поиск минимума функции Розенброка с $\alpha$ = 1 при $\varepsilon = 1e-06$, начальной точке (-6.0, 2.0) методом БФШ}
	        \vspace*{-1.2cm}
            \end{figure}
            
            \begin{figure}[H]
	        \centering
	        \includegraphics[width=0.85\textwidth]{Метод Пауэлла, eps 1e-06, start = (-6.000000, 2.000000), Функция Розенброка с alpha = 1}%
	        \caption{Поиск минимума функции Розенброка с $\alpha$ = 1 при $\varepsilon = 1e-06$, начальной точке (-6.0, 2.0) методом Пауэлла}
	        \vspace*{-1.2cm}
            \end{figure}
            
            \begin{figure}[H]
	        \centering
	        \includegraphics[width=0.85\textwidth]{Метод Давидона-Флетчера-Пауэлла, eps 1e-06, start = (20.000000, -30.000000), Функция Розенброка с alpha = 1}%
	        \caption{Поиск минимума функции Розенброка с $\alpha$ = 1 при $\varepsilon = 1e-06$, начальной точке (20.0, -30.0) методом ДФП}
	        \vspace*{-1.2cm}
            \end{figure}
            
            \begin{figure}[H]
	        \centering
	        \includegraphics[width=0.85\textwidth]{Метод Бройдена-Флетчера-Шенно, eps 1e-06, start = (20.000000, -30.000000), Функция Розенброка с alpha = 1}%
	        \caption{Поиск минимума функции Розенброка с $\alpha$ = 1 при $\varepsilon = 1e-06$, начальной точке (20.0, -30.0) методом БФШ}
	        \vspace*{-1.2cm}
            \end{figure}
            
            \begin{figure}[H]
	        \centering
	        \includegraphics[width=0.85\textwidth]{Метод Пауэлла, eps 1e-06, start = (20.000000, -30.000000), Функция Розенброка с alpha = 1}%
	        \caption{Поиск минимума функции Розенброка с $\alpha$ = 1 при $\varepsilon = 1e-06$, начальной точке (20.0, -30.0) методом Пауэлла}
	        \vspace*{-1.2cm}
            \end{figure}
            \subsubsection{Функция Розенброка с $\alpha$ = 10}

\begin{table}[H]
        \centering
        \vspace*{-1.5em}
        \caption{Результаты работы алгоритмов\\для функции Розенброка с $\alpha$ = 10}
        \footnotesize
        \begin{tabular}{|c|c|c|c|c|}
        \hline
        & &\makecell{Метод ДФП} &\makecell{Метод БФШ} &\makecell{Метод\\Пауэлла} \\
        \hline
	\multirow{10}{*}{\rotatebox[origin=c]{90}{$\varepsilon = 0.01$}}&\textbf{Начальная точка} &\multicolumn{3}{c|}{\textbf{(-6.00, 2.00)}}\\
	\cline{2-5}
	&Точка минимума &(1.01, 1.02) &(1.01, 1.01) &(1.01, 1.01) \\ 
	\cline{2-5}
	&Минимум &0.00 &0.00 &0.00 \\ 
	\cline{2-5}
	&Кол-во итераций &15 &15 &15 \\ 
	\cline{2-5}
	&\makecell{Кол-во вызовов\\целевой функции} &568 &568 &568 \\ 
	\cline{2-5}
	&\makecell{Кол-во вычислений\\градиента} &16 &16 &16 \\ 
	\cline{2-5}
\cline{2-5}&\textbf{Начальная точка} &\multicolumn{3}{c|}{\textbf{(20.00, -30.00)}}\\
	\cline{2-5}
	&Точка минимума &(0.99, 0.98) &(0.99, 0.98) &(0.99, 0.99) \\ 
	\cline{2-5}
	&Минимум &0.00 &0.00 &0.00 \\ 
	\cline{2-5}
	&Кол-во итераций &15 &14 &12 \\ 
	\cline{2-5}
	&\makecell{Кол-во вызовов\\целевой функции} &565 &530 &450 \\ 
	\cline{2-5}
	&\makecell{Кол-во вычислений\\градиента} &16 &15 &13 \\ 
	\cline{2-5}
	\hline
	\multirow{10}{*}{\rotatebox[origin=c]{90}{$\varepsilon = 1e-06$}}&\textbf{Начальная точка} &\multicolumn{3}{c|}{\textbf{(-6.000000, 2.000000)}}\\
	\cline{2-5}
	&Точка минимума &(1.000000, 1.000001) &(1.000001, 1.000002) &(1.000000, 1.000001) \\ 
	\cline{2-5}
	&Минимум &0.000000 &0.000000 &0.000000 \\ 
	\cline{2-5}
	&Кол-во итераций &39 &34 &39 \\ 
	\cline{2-5}
	&\makecell{Кол-во вызовов\\целевой функции} &2185 &1912 &2186 \\ 
	\cline{2-5}
	&\makecell{Кол-во вычислений\\градиента} &40 &35 &40 \\ 
	\cline{2-5}
\cline{2-5}&\textbf{Начальная точка} &\multicolumn{3}{c|}{\textbf{(20.000000, -30.000000)}}\\
	\cline{2-5}
	&Точка минимума &(0.999999, 0.999999) &(0.999999, 0.999998) &(0.999999, 0.999999) \\ 
	\cline{2-5}
	&Минимум &0.000000 &0.000000 &0.000000 \\ 
	\cline{2-5}
	&Кол-во итераций &39 &21 &37 \\ 
	\cline{2-5}
	&\makecell{Кол-во вызовов\\целевой функции} &2183 &1190 &2070 \\ 
	\cline{2-5}
	&\makecell{Кол-во вычислений\\градиента} &40 &22 &38 \\ 
	\cline{2-5}
	\hline

\end{tabular}
\end{table}


            \begin{figure}[H]
	        \centering
	        \includegraphics[width=0.85\textwidth]{Метод Давидона-Флетчера-Пауэлла, eps 0.01, start = (-6.00, 2.00), Функция Розенброка с alpha = 10}%
	        \caption{Поиск минимума функции Розенброка с $\alpha$ = 10 при $\varepsilon = 0.01$, начальной точке (-6.0, 2.0) методом ДФП}
	        \vspace*{-1.2cm}
            \end{figure}
            
            \begin{figure}[H]
	        \centering
	        \includegraphics[width=0.85\textwidth]{Метод Бройдена-Флетчера-Шенно, eps 0.01, start = (-6.00, 2.00), Функция Розенброка с alpha = 10}%
	        \caption{Поиск минимума функции Розенброка с $\alpha$ = 10 при $\varepsilon = 0.01$, начальной точке (-6.0, 2.0) методом БФШ}
	        \vspace*{-1.2cm}
            \end{figure}
            
            \begin{figure}[H]
	        \centering
	        \includegraphics[width=0.85\textwidth]{Метод Пауэлла, eps 0.01, start = (-6.00, 2.00), Функция Розенброка с alpha = 10}%
	        \caption{Поиск минимума функции Розенброка с $\alpha$ = 10 при $\varepsilon = 0.01$, начальной точке (-6.0, 2.0) методом Пауэлла}
	        \vspace*{-1.2cm}
            \end{figure}
            
            \begin{figure}[H]
	        \centering
	        \includegraphics[width=0.85\textwidth]{Метод Давидона-Флетчера-Пауэлла, eps 0.01, start = (20.00, -30.00), Функция Розенброка с alpha = 10}%
	        \caption{Поиск минимума функции Розенброка с $\alpha$ = 10 при $\varepsilon = 0.01$, начальной точке (20.0, -30.0) методом ДФП}
	        \vspace*{-1.2cm}
            \end{figure}
            
            \begin{figure}[H]
	        \centering
	        \includegraphics[width=0.85\textwidth]{Метод Бройдена-Флетчера-Шенно, eps 0.01, start = (20.00, -30.00), Функция Розенброка с alpha = 10}%
	        \caption{Поиск минимума функции Розенброка с $\alpha$ = 10 при $\varepsilon = 0.01$, начальной точке (20.0, -30.0) методом БФШ}
	        \vspace*{-1.2cm}
            \end{figure}
            
            \begin{figure}[H]
	        \centering
	        \includegraphics[width=0.85\textwidth]{Метод Пауэлла, eps 0.01, start = (20.00, -30.00), Функция Розенброка с alpha = 10}%
	        \caption{Поиск минимума функции Розенброка с $\alpha$ = 10 при $\varepsilon = 0.01$, начальной точке (20.0, -30.0) методом Пауэлла}
	        \vspace*{-1.2cm}
            \end{figure}
            
            \begin{figure}[H]
	        \centering
	        \includegraphics[width=0.85\textwidth]{Метод Давидона-Флетчера-Пауэлла, eps 1e-06, start = (-6.000000, 2.000000), Функция Розенброка с alpha = 10}%
	        \caption{Поиск минимума функции Розенброка с $\alpha$ = 10 при $\varepsilon = 1e-06$, начальной точке (-6.0, 2.0) методом ДФП}
	        \vspace*{-1.2cm}
            \end{figure}
            
            \begin{figure}[H]
	        \centering
	        \includegraphics[width=0.85\textwidth]{Метод Бройдена-Флетчера-Шенно, eps 1e-06, start = (-6.000000, 2.000000), Функция Розенброка с alpha = 10}%
	        \caption{Поиск минимума функции Розенброка с $\alpha$ = 10 при $\varepsilon = 1e-06$, начальной точке (-6.0, 2.0) методом БФШ}
	        \vspace*{-1.2cm}
            \end{figure}
            
            \begin{figure}[H]
	        \centering
	        \includegraphics[width=0.85\textwidth]{Метод Пауэлла, eps 1e-06, start = (-6.000000, 2.000000), Функция Розенброка с alpha = 10}%
	        \caption{Поиск минимума функции Розенброка с $\alpha$ = 10 при $\varepsilon = 1e-06$, начальной точке (-6.0, 2.0) методом Пауэлла}
	        \vspace*{-1.2cm}
            \end{figure}
            
            \begin{figure}[H]
	        \centering
	        \includegraphics[width=0.85\textwidth]{Метод Давидона-Флетчера-Пауэлла, eps 1e-06, start = (20.000000, -30.000000), Функция Розенброка с alpha = 10}%
	        \caption{Поиск минимума функции Розенброка с $\alpha$ = 10 при $\varepsilon = 1e-06$, начальной точке (20.0, -30.0) методом ДФП}
	        \vspace*{-1.2cm}
            \end{figure}
            
            \begin{figure}[H]
	        \centering
	        \includegraphics[width=0.85\textwidth]{Метод Бройдена-Флетчера-Шенно, eps 1e-06, start = (20.000000, -30.000000), Функция Розенброка с alpha = 10}%
	        \caption{Поиск минимума функции Розенброка с $\alpha$ = 10 при $\varepsilon = 1e-06$, начальной точке (20.0, -30.0) методом БФШ}
	        \vspace*{-1.2cm}
            \end{figure}
            
            \begin{figure}[H]
	        \centering
	        \includegraphics[width=0.85\textwidth]{Метод Пауэлла, eps 1e-06, start = (20.000000, -30.000000), Функция Розенброка с alpha = 10}%
	        \caption{Поиск минимума функции Розенброка с $\alpha$ = 10 при $\varepsilon = 1e-06$, начальной точке (20.0, -30.0) методом Пауэлла}
	        \vspace*{-1.2cm}
            \end{figure}
            