\subsubsection{Квадратичная функция}

\begin{table}[H]
        \centering
        \vspace*{-1.5em}
        \caption{Результаты работы алгоритмов\\для квадратичной функции}
        \footnotesize
        \begin{tabular}{|c|c|c|c|}
        \hline
        & &\makecell{Метод\\регул. симплекса} &\makecell{Метод\\Нелдера---Мида} \\
        \hline
	\multirow{8}{*}{\rotatebox[origin=c]{90}{$\varepsilon = 0.01$}}&\textbf{Начальная точка} &\multicolumn{2}{c|}{\textbf{(-6.00, 2.00)}}\\
	\cline{2-4}
	&Точка минимума &(2.24, -0.00) &(2.25, 0.01) \\ 
	\cline{2-4}
	&Минимум &-66.00 &-66.00 \\ 
	\cline{2-4}
	&Кол-во итераций &19 &25 \\ 
	\cline{2-4}
	&\makecell{Кол-во вызовов\\целевой функции} &39 &76 \\ 
	\cline{2-4}
\cline{2-4}&\textbf{Начальная точка} &\multicolumn{2}{c|}{\textbf{(20.00, -30.00)}}\\
	\cline{2-4}
	&Точка минимума &(2.23, -0.00) &(2.23, -0.01) \\ 
	\cline{2-4}
	&Минимум &-66.00 &-66.00 \\ 
	\cline{2-4}
	&Кол-во итераций &46 &31 \\ 
	\cline{2-4}
	&\makecell{Кол-во вызовов\\целевой функции} &66 &92 \\ 
	\cline{2-4}
	\hline
	\multirow{8}{*}{\rotatebox[origin=c]{90}{$\varepsilon = 1e-06$}}&\textbf{Начальная точка} &\multicolumn{2}{c|}{\textbf{(-6.000000, 2.000000)}}\\
	\cline{2-4}
	&Точка минимума &(2.236068, 0.000000) &(2.236079, -0.000007) \\ 
	\cline{2-4}
	&Минимум &-66.000000 &-66.000000 \\ 
	\cline{2-4}
	&Кол-во итераций &35 &40 \\ 
	\cline{2-4}
	&\makecell{Кол-во вызовов\\целевой функции} &81 &120 \\ 
	\cline{2-4}
\cline{2-4}&\textbf{Начальная точка} &\multicolumn{2}{c|}{\textbf{(20.000000, -30.000000)}}\\
	\cline{2-4}
	&Точка минимума &(2.236068, 0.000000) &(2.236059, -0.000002) \\ 
	\cline{2-4}
	&Минимум &-66.000000 &-66.000000 \\ 
	\cline{2-4}
	&Кол-во итераций &63 &46 \\ 
	\cline{2-4}
	&\makecell{Кол-во вызовов\\целевой функции} &109 &135 \\ 
	\cline{2-4}
	\hline

\end{tabular}
\end{table}


            \begin{figure}[H]
	        \centering
	        \includegraphics[width=0.80\textwidth]{Регулярный симплекс, eps 0.01, start = (-6.00, 2.00), Квадратичная функция}%
	        \caption{Поиск минимума квадратичной функции при $\varepsilon = 0.01$, начальной точке (-6.0, 2.0) методом регулярного симплекса}
	        \vspace*{-1.2cm}
            \end{figure}
            
            \begin{figure}[H]
	        \centering
	        \includegraphics[width=0.80\textwidth]{Метод Нелдера-Мида, eps 0.01, start = (-6.00, 2.00), Квадратичная функция}%
	        \caption{Поиск минимума квадратичной функции при $\varepsilon = 0.01$, начальной точке (-6.0, 2.0) методом Нелдера---Мида}
	        \vspace*{-1.2cm}
            \end{figure}
            
            \begin{figure}[H]
	        \centering
	        \includegraphics[width=0.80\textwidth]{Регулярный симплекс, eps 0.01, start = (20.00, -30.00), Квадратичная функция}%
	        \caption{Поиск минимума квадратичной функции при $\varepsilon = 0.01$, начальной точке (20.0, -30.0) методом регулярного симплекса}
	        \vspace*{-1.2cm}
            \end{figure}
            
            \begin{figure}[H]
	        \centering
	        \includegraphics[width=0.80\textwidth]{Метод Нелдера-Мида, eps 0.01, start = (20.00, -30.00), Квадратичная функция}%
	        \caption{Поиск минимума квадратичной функции при $\varepsilon = 0.01$, начальной точке (20.0, -30.0) методом Нелдера---Мида}
	        \vspace*{-1.2cm}
            \end{figure}
            
            \begin{figure}[H]
	        \centering
	        \includegraphics[width=0.80\textwidth]{Регулярный симплекс, eps 1e-06, start = (-6.000000, 2.000000), Квадратичная функция}%
	        \caption{Поиск минимума квадратичной функции при $\varepsilon = 1e-06$, начальной точке (-6.0, 2.0) методом регулярного симплекса}
	        \vspace*{-1.2cm}
            \end{figure}
            
            \begin{figure}[H]
	        \centering
	        \includegraphics[width=0.80\textwidth]{Метод Нелдера-Мида, eps 1e-06, start = (-6.000000, 2.000000), Квадратичная функция}%
	        \caption{Поиск минимума квадратичной функции при $\varepsilon = 1e-06$, начальной точке (-6.0, 2.0) методом Нелдера---Мида}
	        \vspace*{-1.2cm}
            \end{figure}
            
            \begin{figure}[H]
	        \centering
	        \includegraphics[width=0.80\textwidth]{Регулярный симплекс, eps 1e-06, start = (20.000000, -30.000000), Квадратичная функция}%
	        \caption{Поиск минимума квадратичной функции при $\varepsilon = 1e-06$, начальной точке (20.0, -30.0) методом регулярного симплекса}
	        \vspace*{-1.2cm}
            \end{figure}
            
            \begin{figure}[H]
	        \centering
	        \includegraphics[width=0.80\textwidth]{Метод Нелдера-Мида, eps 1e-06, start = (20.000000, -30.000000), Квадратичная функция}%
	        \caption{Поиск минимума квадратичной функции при $\varepsilon = 1e-06$, начальной точке (20.0, -30.0) методом Нелдера---Мида}
	        \vspace*{-1.2cm}
            \end{figure}
            \subsubsection{Функция Розенброка с $\alpha$ = 1}

\begin{table}[H]
        \centering
        \vspace*{-1.5em}
        \caption{Результаты работы алгоритмов\\для функции Розенброка с $\alpha$ = 1}
        \footnotesize
        \begin{tabular}{|c|c|c|c|}
        \hline
        & &\makecell{Метод\\регул. симплекса} &\makecell{Метод\\Нелдера---Мида} \\
        \hline
	\multirow{8}{*}{\rotatebox[origin=c]{90}{$\varepsilon = 0.01$}}&\textbf{Начальная точка} &\multicolumn{2}{c|}{\textbf{(-6.00, 2.00)}}\\
	\cline{2-4}
	&Точка минимума &(0.97, 0.93) &(0.99, 0.97) \\ 
	\cline{2-4}
	&Минимум &0.00 &0.00 \\ 
	\cline{2-4}
	&Кол-во итераций &34 &23 \\ 
	\cline{2-4}
	&\makecell{Кол-во вызовов\\целевой функции} &54 &71 \\ 
	\cline{2-4}
\cline{2-4}&\textbf{Начальная точка} &\multicolumn{2}{c|}{\textbf{(20.00, -30.00)}}\\
	\cline{2-4}
	&Точка минимума &(0.92, 0.81) &(0.99, 0.97) \\ 
	\cline{2-4}
	&Минимум &0.01 &0.00 \\ 
	\cline{2-4}
	&Кол-во итераций &124 &34 \\ 
	\cline{2-4}
	&\makecell{Кол-во вызовов\\целевой функции} &144 &100 \\ 
	\cline{2-4}
	\hline
	\multirow{8}{*}{\rotatebox[origin=c]{90}{$\varepsilon = 1e-06$}}&\textbf{Начальная точка} &\multicolumn{2}{c|}{\textbf{(-6.000000, 2.000000)}}\\
	\cline{2-4}
	&Точка минимума &(0.999986, 0.999965) &(1.000158, 1.000429) \\ 
	\cline{2-4}
	&Минимум &0.000000 &0.000000 \\ 
	\cline{2-4}
	&Кол-во итераций &461 &40 \\ 
	\cline{2-4}
	&\makecell{Кол-во вызовов\\целевой функции} &507 &120 \\ 
	\cline{2-4}
\cline{2-4}&\textbf{Начальная точка} &\multicolumn{2}{c|}{\textbf{(20.000000, -30.000000)}}\\
	\cline{2-4}
	&Точка минимума &(0.999985, 0.999965) &(0.999845, 0.999916) \\ 
	\cline{2-4}
	&Минимум &0.000000 &0.000000 \\ 
	\cline{2-4}
	&Кол-во итераций &632 &51 \\ 
	\cline{2-4}
	&\makecell{Кол-во вызовов\\целевой функции} &678 &147 \\ 
	\cline{2-4}
	\hline

\end{tabular}
\end{table}


            \begin{figure}[H]
	        \centering
	        \includegraphics[width=0.80\textwidth]{Регулярный симплекс, eps 0.01, start = (-6.00, 2.00), Функция Розенброка с alpha = 1}%
	        \caption{Поиск минимума функции Розенброка с $\alpha$ = 1 при $\varepsilon = 0.01$, начальной точке (-6.0, 2.0) методом регулярного симплекса}
	        \vspace*{-1.2cm}
            \end{figure}
            
            \begin{figure}[H]
	        \centering
	        \includegraphics[width=0.80\textwidth]{Метод Нелдера-Мида, eps 0.01, start = (-6.00, 2.00), Функция Розенброка с alpha = 1}%
	        \caption{Поиск минимума функции Розенброка с $\alpha$ = 1 при $\varepsilon = 0.01$, начальной точке (-6.0, 2.0) методом Нелдера---Мида}
	        \vspace*{-1.2cm}
            \end{figure}
            
            \begin{figure}[H]
	        \centering
	        \includegraphics[width=0.80\textwidth]{Регулярный симплекс, eps 0.01, start = (20.00, -30.00), Функция Розенброка с alpha = 1}%
	        \caption{Поиск минимума функции Розенброка с $\alpha$ = 1 при $\varepsilon = 0.01$, начальной точке (20.0, -30.0) методом регулярного симплекса}
	        \vspace*{-1.2cm}
            \end{figure}
            
            \begin{figure}[H]
	        \centering
	        \includegraphics[width=0.80\textwidth]{Метод Нелдера-Мида, eps 0.01, start = (20.00, -30.00), Функция Розенброка с alpha = 1}%
	        \caption{Поиск минимума функции Розенброка с $\alpha$ = 1 при $\varepsilon = 0.01$, начальной точке (20.0, -30.0) методом Нелдера---Мида}
	        \vspace*{-1.2cm}
            \end{figure}
            
            \begin{figure}[H]
	        \centering
	        \includegraphics[width=0.80\textwidth]{Регулярный симплекс, eps 1e-06, start = (-6.000000, 2.000000), Функция Розенброка с alpha = 1}%
	        \caption{Поиск минимума функции Розенброка с $\alpha$ = 1 при $\varepsilon = 1e-06$, начальной точке (-6.0, 2.0) методом регулярного симплекса}
	        \vspace*{-1.2cm}
            \end{figure}
            
            \begin{figure}[H]
	        \centering
	        \includegraphics[width=0.80\textwidth]{Метод Нелдера-Мида, eps 1e-06, start = (-6.000000, 2.000000), Функция Розенброка с alpha = 1}%
	        \caption{Поиск минимума функции Розенброка с $\alpha$ = 1 при $\varepsilon = 1e-06$, начальной точке (-6.0, 2.0) методом Нелдера---Мида}
	        \vspace*{-1.2cm}
            \end{figure}
            
            \begin{figure}[H]
	        \centering
	        \includegraphics[width=0.80\textwidth]{Регулярный симплекс, eps 1e-06, start = (20.000000, -30.000000), Функция Розенброка с alpha = 1}%
	        \caption{Поиск минимума функции Розенброка с $\alpha$ = 1 при $\varepsilon = 1e-06$, начальной точке (20.0, -30.0) методом регулярного симплекса}
	        \vspace*{-1.2cm}
            \end{figure}
            
            \begin{figure}[H]
	        \centering
	        \includegraphics[width=0.80\textwidth]{Метод Нелдера-Мида, eps 1e-06, start = (20.000000, -30.000000), Функция Розенброка с alpha = 1}%
	        \caption{Поиск минимума функции Розенброка с $\alpha$ = 1 при $\varepsilon = 1e-06$, начальной точке (20.0, -30.0) методом Нелдера---Мида}
	        \vspace*{-1.2cm}
            \end{figure}
            \subsubsection{Функция Розенброка с $\alpha$ = 10}

\begin{table}[H]
        \centering
        \vspace*{-1.5em}
        \caption{Результаты работы алгоритмов\\для функции Розенброка с $\alpha$ = 10}
        \footnotesize
        \begin{tabular}{|c|c|c|c|}
        \hline
        & &\makecell{Метод\\регул. симплекса} &\makecell{Метод\\Нелдера---Мида} \\
        \hline
	\multirow{8}{*}{\rotatebox[origin=c]{90}{$\varepsilon = 0.01$}}&\textbf{Начальная точка} &\multicolumn{2}{c|}{\textbf{(-6.00, 2.00)}}\\
	\cline{2-4}
	&Точка минимума &(0.63, 0.38) &(0.98, 0.97) \\ 
	\cline{2-4}
	&Минимум &0.14 &0.00 \\ 
	\cline{2-4}
	&Кол-во итераций &157 &43 \\ 
	\cline{2-4}
	&\makecell{Кол-во вызовов\\целевой функции} &177 &121 \\ 
	\cline{2-4}
\cline{2-4}&\textbf{Начальная точка} &\multicolumn{2}{c|}{\textbf{(20.00, -30.00)}}\\
	\cline{2-4}
	&Точка минимума &(0.78, 0.60) &(1.00, 1.01) \\ 
	\cline{2-4}
	&Минимум &0.05 &0.00 \\ 
	\cline{2-4}
	&Кол-во итераций &99 &37 \\ 
	\cline{2-4}
	&\makecell{Кол-во вызовов\\целевой функции} &119 &108 \\ 
	\cline{2-4}
	\hline
	\multirow{8}{*}{\rotatebox[origin=c]{90}{$\varepsilon = 1e-06$}}&\textbf{Начальная точка} &\multicolumn{2}{c|}{\textbf{(-6.000000, 2.000000)}}\\
	\cline{2-4}
	&Точка минимума &(0.999851, 0.999696) &(1.000015, 1.000041) \\ 
	\cline{2-4}
	&Минимум &0.000000 &0.000000 \\ 
	\cline{2-4}
	&Кол-во итераций &3673 &61 \\ 
	\cline{2-4}
	&\makecell{Кол-во вызовов\\целевой функции} &3719 &174 \\ 
	\cline{2-4}
\cline{2-4}&\textbf{Начальная точка} &\multicolumn{2}{c|}{\textbf{(20.000000, -30.000000)}}\\
	\cline{2-4}
	&Точка минимума &(0.999849, 0.999692) &(0.999911, 0.999808) \\ 
	\cline{2-4}
	&Минимум &0.000000 &0.000000 \\ 
	\cline{2-4}
	&Кол-во итераций &3545 &50 \\ 
	\cline{2-4}
	&\makecell{Кол-во вызовов\\целевой функции} &3591 &146 \\ 
	\cline{2-4}
	\hline

\end{tabular}
\end{table}


            \begin{figure}[H]
	        \centering
	        \includegraphics[width=0.80\textwidth]{Регулярный симплекс, eps 0.01, start = (-6.00, 2.00), Функция Розенброка с alpha = 10}%
	        \caption{Поиск минимума функции Розенброка с $\alpha$ = 10 при $\varepsilon = 0.01$, начальной точке (-6.0, 2.0) методом регулярного симплекса}
	        \vspace*{-1.2cm}
            \end{figure}
            
            \begin{figure}[H]
	        \centering
	        \includegraphics[width=0.80\textwidth]{Метод Нелдера-Мида, eps 0.01, start = (-6.00, 2.00), Функция Розенброка с alpha = 10}%
	        \caption{Поиск минимума функции Розенброка с $\alpha$ = 10 при $\varepsilon = 0.01$, начальной точке (-6.0, 2.0) методом Нелдера---Мида}
	        \vspace*{-1.2cm}
            \end{figure}
            
            \begin{figure}[H]
	        \centering
	        \includegraphics[width=0.80\textwidth]{Регулярный симплекс, eps 0.01, start = (20.00, -30.00), Функция Розенброка с alpha = 10}%
	        \caption{Поиск минимума функции Розенброка с $\alpha$ = 10 при $\varepsilon = 0.01$, начальной точке (20.0, -30.0) методом регулярного симплекса}
	        \vspace*{-1.2cm}
            \end{figure}
            
            \begin{figure}[H]
	        \centering
	        \includegraphics[width=0.80\textwidth]{Метод Нелдера-Мида, eps 0.01, start = (20.00, -30.00), Функция Розенброка с alpha = 10}%
	        \caption{Поиск минимума функции Розенброка с $\alpha$ = 10 при $\varepsilon = 0.01$, начальной точке (20.0, -30.0) методом Нелдера---Мида}
	        \vspace*{-1.2cm}
            \end{figure}
            
            \begin{figure}[H]
	        \centering
	        \includegraphics[width=0.80\textwidth]{Регулярный симплекс, eps 1e-06, start = (-6.000000, 2.000000), Функция Розенброка с alpha = 10}%
	        \caption{Поиск минимума функции Розенброка с $\alpha$ = 10 при $\varepsilon = 1e-06$, начальной точке (-6.0, 2.0) методом регулярного симплекса}
	        \vspace*{-1.2cm}
            \end{figure}
            
            \begin{figure}[H]
	        \centering
	        \includegraphics[width=0.80\textwidth]{Метод Нелдера-Мида, eps 1e-06, start = (-6.000000, 2.000000), Функция Розенброка с alpha = 10}%
	        \caption{Поиск минимума функции Розенброка с $\alpha$ = 10 при $\varepsilon = 1e-06$, начальной точке (-6.0, 2.0) методом Нелдера---Мида}
	        \vspace*{-1.2cm}
            \end{figure}
            
            \begin{figure}[H]
	        \centering
	        \includegraphics[width=0.80\textwidth]{Регулярный симплекс, eps 1e-06, start = (20.000000, -30.000000), Функция Розенброка с alpha = 10}%
	        \caption{Поиск минимума функции Розенброка с $\alpha$ = 10 при $\varepsilon = 1e-06$, начальной точке (20.0, -30.0) методом регулярного симплекса}
	        \vspace*{-1.2cm}
            \end{figure}
            
            \begin{figure}[H]
	        \centering
	        \includegraphics[width=0.80\textwidth]{Метод Нелдера-Мида, eps 1e-06, start = (20.000000, -30.000000), Функция Розенброка с alpha = 10}%
	        \caption{Поиск минимума функции Розенброка с $\alpha$ = 10 при $\varepsilon = 1e-06$, начальной точке (20.0, -30.0) методом Нелдера---Мида}
	        \vspace*{-1.2cm}
            \end{figure}
            