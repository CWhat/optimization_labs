\subsection{Квадратичная функция}

\begin{table}[H]
        \centering
        \vspace*{-1.5em}
        \caption{Результаты работы алгоритмов\\для квадратичной функции}
        \footnotesize
        \begin{tabular}{|c|c|c|c|c|}
        \hline
        & &\makecell{Метод\\Ньютона} &\makecell{Модифицир.\\метод Ньютона} &\makecell{Метод\\Марквардта} \\
        \hline
	\multirow{12}{*}{\rotatebox[origin=c]{90}{$\varepsilon = 0.01$}}&\textbf{Начальная точка} &\multicolumn{3}{c|}{\textbf{(-6.00, 2.00)}}\\
	\cline{2-5}
	&Точка минимума &(2.24, 0.00) &(2.24, -0.00) &(2.24, -0.00) \\ 
	\cline{2-5}
	&Минимум &-66.00 &-66.00 &-66.00 \\ 
	\cline{2-5}
	&Кол-во итераций &1 &2 &8 \\ 
	\cline{2-5}
	&\makecell{Кол-во вызовов\\целевой функции} &0 &44 &9 \\ 
	\cline{2-5}
	&\makecell{Кол-во вычислений\\градиента} &2 &3 &9 \\ 
	\cline{2-5}
	&\makecell{Кол-во вычислений\\матриц Гессе} &1 &2 &8 \\ 
	\cline{2-5}
\cline{2-5}&\textbf{Начальная точка} &\multicolumn{3}{c|}{\textbf{(20.00, -30.00)}}\\
	\cline{2-5}
	&Точка минимума &(2.24, 0.00) &(2.24, -0.00) &(2.24, -0.00) \\ 
	\cline{2-5}
	&Минимум &-66.00 &-66.00 &-66.00 \\ 
	\cline{2-5}
	&Кол-во итераций &1 &2 &9 \\ 
	\cline{2-5}
	&\makecell{Кол-во вызовов\\целевой функции} &0 &44 &10 \\ 
	\cline{2-5}
	&\makecell{Кол-во вычислений\\градиента} &2 &3 &10 \\ 
	\cline{2-5}
	&\makecell{Кол-во вычислений\\матриц Гессе} &1 &2 &9 \\ 
	\cline{2-5}
	\hline
	\multirow{12}{*}{\rotatebox[origin=c]{90}{$\varepsilon = 1e-06$}}&\textbf{Начальная точка} &\multicolumn{3}{c|}{\textbf{(-6.000000, 2.000000)}}\\
	\cline{2-5}
	&Точка минимума &(2.236068, 0.000000) &(2.236068, -0.000000) &(2.236068, -0.000000) \\ 
	\cline{2-5}
	&Минимум &-66.000000 &-66.000000 &-66.000000 \\ 
	\cline{2-5}
	&Кол-во итераций &1 &2 &11 \\ 
	\cline{2-5}
	&\makecell{Кол-во вызовов\\целевой функции} &0 &82 &12 \\ 
	\cline{2-5}
	&\makecell{Кол-во вычислений\\градиента} &2 &3 &12 \\ 
	\cline{2-5}
	&\makecell{Кол-во вычислений\\матриц Гессе} &1 &2 &11 \\ 
	\cline{2-5}
\cline{2-5}&\textbf{Начальная точка} &\multicolumn{3}{c|}{\textbf{(20.000000, -30.000000)}}\\
	\cline{2-5}
	&Точка минимума &(2.236068, 0.000000) &(2.236068, -0.000000) &(2.236068, -0.000000) \\ 
	\cline{2-5}
	&Минимум &-66.000000 &-66.000000 &-66.000000 \\ 
	\cline{2-5}
	&Кол-во итераций &1 &2 &11 \\ 
	\cline{2-5}
	&\makecell{Кол-во вызовов\\целевой функции} &0 &82 &12 \\ 
	\cline{2-5}
	&\makecell{Кол-во вычислений\\градиента} &2 &3 &12 \\ 
	\cline{2-5}
	&\makecell{Кол-во вычислений\\матриц Гессе} &1 &2 &11 \\ 
	\cline{2-5}
	\hline

\end{tabular}
\end{table}


            \begin{figure}[H]
	        \centering
	        \includegraphics[width=0.70\textwidth]{Классический метод Ньютона, eps 0.01, start = (-6.00, 2.00), Квадратичная функция}%
	        \caption{Поиск минимума квадратичной функции при $\varepsilon = 0.01$, начальной точке (-6.0, 2.0) классическим методом Ньютона}
	        \vspace*{-1.2cm}
            \end{figure}
            
            \begin{figure}[H]
	        \centering
	        \includegraphics[width=0.70\textwidth]{Модификация метода Ньютона с наискорейшим спуском, eps 0.01, start = (-6.00, 2.00), Квадратичная функция}%
	        \caption{Поиск минимума квадратичной функции при $\varepsilon = 0.01$, начальной точке (-6.0, 2.0) методом Ньютона с наискорейшим спуском}
	        \vspace*{-1.2cm}
            \end{figure}
            
            \begin{figure}[H]
	        \centering
	        \includegraphics[width=0.70\textwidth]{Метод Марквардта, eps 0.01, start = (-6.00, 2.00), Квадратичная функция}%
	        \caption{Поиск минимума квадратичной функции при $\varepsilon = 0.01$, начальной точке (-6.0, 2.0) методом Марквардта}
	        \vspace*{-1.2cm}
            \end{figure}
            
            \begin{figure}[H]
	        \centering
	        \includegraphics[width=0.70\textwidth]{Классический метод Ньютона, eps 0.01, start = (20.00, -30.00), Квадратичная функция}%
	        \caption{Поиск минимума квадратичной функции при $\varepsilon = 0.01$, начальной точке (20.0, -30.0) классическим методом Ньютона}
	        \vspace*{-1.2cm}
            \end{figure}
            
            \begin{figure}[H]
	        \centering
	        \includegraphics[width=0.70\textwidth]{Модификация метода Ньютона с наискорейшим спуском, eps 0.01, start = (20.00, -30.00), Квадратичная функция}%
	        \caption{Поиск минимума квадратичной функции при $\varepsilon = 0.01$, начальной точке (20.0, -30.0) методом Ньютона с наискорейшим спуском}
	        \vspace*{-1.2cm}
            \end{figure}
            
            \begin{figure}[H]
	        \centering
	        \includegraphics[width=0.70\textwidth]{Метод Марквардта, eps 0.01, start = (20.00, -30.00), Квадратичная функция}%
	        \caption{Поиск минимума квадратичной функции при $\varepsilon = 0.01$, начальной точке (20.0, -30.0) методом Марквардта}
	        \vspace*{-1.2cm}
            \end{figure}
            
            \begin{figure}[H]
	        \centering
	        \includegraphics[width=0.70\textwidth]{Классический метод Ньютона, eps 1e-06, start = (-6.000000, 2.000000), Квадратичная функция}%
	        \caption{Поиск минимума квадратичной функции при $\varepsilon = 1e-06$, начальной точке (-6.0, 2.0) классическим методом Ньютона}
	        \vspace*{-1.2cm}
            \end{figure}
            
            \begin{figure}[H]
	        \centering
	        \includegraphics[width=0.70\textwidth]{Модификация метода Ньютона с наискорейшим спуском, eps 1e-06, start = (-6.000000, 2.000000), Квадратичная функция}%
	        \caption{Поиск минимума квадратичной функции при $\varepsilon = 1e-06$, начальной точке (-6.0, 2.0) методом Ньютона с наискорейшим спуском}
	        \vspace*{-1.2cm}
            \end{figure}
            
            \begin{figure}[H]
	        \centering
	        \includegraphics[width=0.70\textwidth]{Метод Марквардта, eps 1e-06, start = (-6.000000, 2.000000), Квадратичная функция}%
	        \caption{Поиск минимума квадратичной функции при $\varepsilon = 1e-06$, начальной точке (-6.0, 2.0) методом Марквардта}
	        \vspace*{-1.2cm}
            \end{figure}
            
            \begin{figure}[H]
	        \centering
	        \includegraphics[width=0.70\textwidth]{Классический метод Ньютона, eps 1e-06, start = (20.000000, -30.000000), Квадратичная функция}%
	        \caption{Поиск минимума квадратичной функции при $\varepsilon = 1e-06$, начальной точке (20.0, -30.0) классическим методом Ньютона}
	        \vspace*{-1.2cm}
            \end{figure}
            
            \begin{figure}[H]
	        \centering
	        \includegraphics[width=0.70\textwidth]{Модификация метода Ньютона с наискорейшим спуском, eps 1e-06, start = (20.000000, -30.000000), Квадратичная функция}%
	        \caption{Поиск минимума квадратичной функции при $\varepsilon = 1e-06$, начальной точке (20.0, -30.0) методом Ньютона с наискорейшим спуском}
	        \vspace*{-1.2cm}
            \end{figure}
            
            \begin{figure}[H]
	        \centering
	        \includegraphics[width=0.70\textwidth]{Метод Марквардта, eps 1e-06, start = (20.000000, -30.000000), Квадратичная функция}%
	        \caption{Поиск минимума квадратичной функции при $\varepsilon = 1e-06$, начальной точке (20.0, -30.0) методом Марквардта}
	        \vspace*{-1.2cm}
            \end{figure}
            \subsection{Функция Розенброка с $\alpha$ = 1}

\begin{table}[H]
        \centering
        \vspace*{-1.5em}
        \caption{Результаты работы алгоритмов\\для функции Розенброка с $\alpha$ = 1}
        \footnotesize
        \begin{tabular}{|c|c|c|c|c|}
        \hline
        & &\makecell{Метод\\Ньютона} &\makecell{Модифицир.\\метод Ньютона} &\makecell{Метод\\Марквардта} \\
        \hline
	\multirow{12}{*}{\rotatebox[origin=c]{90}{$\varepsilon = 0.01$}}&\textbf{Начальная точка} &\multicolumn{3}{c|}{\textbf{(-6.00, 2.00)}}\\
	\cline{2-5}
	&Точка минимума &(1.00, 1.00) &(1.00, 1.00) &(1.00, 1.00) \\ 
	\cline{2-5}
	&Минимум &0.00 &0.00 &0.00 \\ 
	\cline{2-5}
	&Кол-во итераций &4 &11 &13 \\ 
	\cline{2-5}
	&\makecell{Кол-во вызовов\\целевой функции} &0 &233 &14 \\ 
	\cline{2-5}
	&\makecell{Кол-во вычислений\\градиента} &5 &12 &14 \\ 
	\cline{2-5}
	&\makecell{Кол-во вычислений\\матриц Гессе} &4 &11 &13 \\ 
	\cline{2-5}
\cline{2-5}&\textbf{Начальная точка} &\multicolumn{3}{c|}{\textbf{(20.00, -30.00)}}\\
	\cline{2-5}
	&Точка минимума &(1.00, 1.00) &(1.00, 1.00) &(0.99, 0.98) \\ 
	\cline{2-5}
	&Минимум &0.00 &0.00 &0.00 \\ 
	\cline{2-5}
	&Кол-во итераций &4 &11 &12 \\ 
	\cline{2-5}
	&\makecell{Кол-во вызовов\\целевой функции} &0 &233 &13 \\ 
	\cline{2-5}
	&\makecell{Кол-во вычислений\\градиента} &5 &12 &13 \\ 
	\cline{2-5}
	&\makecell{Кол-во вычислений\\матриц Гессе} &4 &11 &12 \\ 
	\cline{2-5}
	\hline
	\multirow{12}{*}{\rotatebox[origin=c]{90}{$\varepsilon = 1e-06$}}&\textbf{Начальная точка} &\multicolumn{3}{c|}{\textbf{(-6.000000, 2.000000)}}\\
	\cline{2-5}
	&Точка минимума &(1.000000, 1.000000) &(1.000000, 1.000000) &(0.999999, 0.999999) \\ 
	\cline{2-5}
	&Минимум &0.000000 &0.000000 &0.000000 \\ 
	\cline{2-5}
	&Кол-во итераций &5 &12 &15 \\ 
	\cline{2-5}
	&\makecell{Кол-во вызовов\\целевой функции} &0 &485 &16 \\ 
	\cline{2-5}
	&\makecell{Кол-во вычислений\\градиента} &6 &13 &16 \\ 
	\cline{2-5}
	&\makecell{Кол-во вычислений\\матриц Гессе} &5 &12 &15 \\ 
	\cline{2-5}
\cline{2-5}&\textbf{Начальная точка} &\multicolumn{3}{c|}{\textbf{(20.000000, -30.000000)}}\\
	\cline{2-5}
	&Точка минимума &(1.000000, 1.000000) &(1.000000, 1.000000) &(1.000000, 0.999999) \\ 
	\cline{2-5}
	&Минимум &0.000000 &0.000000 &0.000000 \\ 
	\cline{2-5}
	&Кол-во итераций &5 &12 &15 \\ 
	\cline{2-5}
	&\makecell{Кол-во вызовов\\целевой функции} &0 &485 &16 \\ 
	\cline{2-5}
	&\makecell{Кол-во вычислений\\градиента} &6 &13 &16 \\ 
	\cline{2-5}
	&\makecell{Кол-во вычислений\\матриц Гессе} &5 &12 &15 \\ 
	\cline{2-5}
	\hline

\end{tabular}
\end{table}


            \begin{figure}[H]
	        \centering
	        \includegraphics[width=0.70\textwidth]{Классический метод Ньютона, eps 0.01, start = (-6.00, 2.00), Функция Розенброка с alpha = 1}%
	        \caption{Поиск минимума функции Розенброка с $\alpha$ = 1 при $\varepsilon = 0.01$, начальной точке (-6.0, 2.0) классическим методом Ньютона}
	        \vspace*{-1.2cm}
            \end{figure}
            
            \begin{figure}[H]
	        \centering
	        \includegraphics[width=0.70\textwidth]{Модификация метода Ньютона с наискорейшим спуском, eps 0.01, start = (-6.00, 2.00), Функция Розенброка с alpha = 1}%
	        \caption{Поиск минимума функции Розенброка с $\alpha$ = 1 при $\varepsilon = 0.01$, начальной точке (-6.0, 2.0) методом Ньютона с наискорейшим спуском}
	        \vspace*{-1.2cm}
            \end{figure}
            
            \begin{figure}[H]
	        \centering
	        \includegraphics[width=0.70\textwidth]{Метод Марквардта, eps 0.01, start = (-6.00, 2.00), Функция Розенброка с alpha = 1}%
	        \caption{Поиск минимума функции Розенброка с $\alpha$ = 1 при $\varepsilon = 0.01$, начальной точке (-6.0, 2.0) методом Марквардта}
	        \vspace*{-1.2cm}
            \end{figure}
            
            \begin{figure}[H]
	        \centering
	        \includegraphics[width=0.70\textwidth]{Классический метод Ньютона, eps 0.01, start = (20.00, -30.00), Функция Розенброка с alpha = 1}%
	        \caption{Поиск минимума функции Розенброка с $\alpha$ = 1 при $\varepsilon = 0.01$, начальной точке (20.0, -30.0) классическим методом Ньютона}
	        \vspace*{-1.2cm}
            \end{figure}
            
            \begin{figure}[H]
	        \centering
	        \includegraphics[width=0.70\textwidth]{Модификация метода Ньютона с наискорейшим спуском, eps 0.01, start = (20.00, -30.00), Функция Розенброка с alpha = 1}%
	        \caption{Поиск минимума функции Розенброка с $\alpha$ = 1 при $\varepsilon = 0.01$, начальной точке (20.0, -30.0) методом Ньютона с наискорейшим спуском}
	        \vspace*{-1.2cm}
            \end{figure}
            
            \begin{figure}[H]
	        \centering
	        \includegraphics[width=0.70\textwidth]{Метод Марквардта, eps 0.01, start = (20.00, -30.00), Функция Розенброка с alpha = 1}%
	        \caption{Поиск минимума функции Розенброка с $\alpha$ = 1 при $\varepsilon = 0.01$, начальной точке (20.0, -30.0) методом Марквардта}
	        \vspace*{-1.2cm}
            \end{figure}
            
            \begin{figure}[H]
	        \centering
	        \includegraphics[width=0.70\textwidth]{Классический метод Ньютона, eps 1e-06, start = (-6.000000, 2.000000), Функция Розенброка с alpha = 1}%
	        \caption{Поиск минимума функции Розенброка с $\alpha$ = 1 при $\varepsilon = 1e-06$, начальной точке (-6.0, 2.0) классическим методом Ньютона}
	        \vspace*{-1.2cm}
            \end{figure}
            
            \begin{figure}[H]
	        \centering
	        \includegraphics[width=0.70\textwidth]{Модификация метода Ньютона с наискорейшим спуском, eps 1e-06, start = (-6.000000, 2.000000), Функция Розенброка с alpha = 1}%
	        \caption{Поиск минимума функции Розенброка с $\alpha$ = 1 при $\varepsilon = 1e-06$, начальной точке (-6.0, 2.0) методом Ньютона с наискорейшим спуском}
	        \vspace*{-1.2cm}
            \end{figure}
            
            \begin{figure}[H]
	        \centering
	        \includegraphics[width=0.70\textwidth]{Метод Марквардта, eps 1e-06, start = (-6.000000, 2.000000), Функция Розенброка с alpha = 1}%
	        \caption{Поиск минимума функции Розенброка с $\alpha$ = 1 при $\varepsilon = 1e-06$, начальной точке (-6.0, 2.0) методом Марквардта}
	        \vspace*{-1.2cm}
            \end{figure}
            
            \begin{figure}[H]
	        \centering
	        \includegraphics[width=0.70\textwidth]{Классический метод Ньютона, eps 1e-06, start = (20.000000, -30.000000), Функция Розенброка с alpha = 1}%
	        \caption{Поиск минимума функции Розенброка с $\alpha$ = 1 при $\varepsilon = 1e-06$, начальной точке (20.0, -30.0) классическим методом Ньютона}
	        \vspace*{-1.2cm}
            \end{figure}
            
            \begin{figure}[H]
	        \centering
	        \includegraphics[width=0.70\textwidth]{Модификация метода Ньютона с наискорейшим спуском, eps 1e-06, start = (20.000000, -30.000000), Функция Розенброка с alpha = 1}%
	        \caption{Поиск минимума функции Розенброка с $\alpha$ = 1 при $\varepsilon = 1e-06$, начальной точке (20.0, -30.0) методом Ньютона с наискорейшим спуском}
	        \vspace*{-1.2cm}
            \end{figure}
            
            \begin{figure}[H]
	        \centering
	        \includegraphics[width=0.70\textwidth]{Метод Марквардта, eps 1e-06, start = (20.000000, -30.000000), Функция Розенброка с alpha = 1}%
	        \caption{Поиск минимума функции Розенброка с $\alpha$ = 1 при $\varepsilon = 1e-06$, начальной точке (20.0, -30.0) методом Марквардта}
	        \vspace*{-1.2cm}
            \end{figure}
            \subsection{Функция Розенброка с $\alpha$ = 10}

\begin{table}[H]
        \centering
        \vspace*{-1.5em}
        \caption{Результаты работы алгоритмов\\для функции Розенброка с $\alpha$ = 10}
        \footnotesize
        \begin{tabular}{|c|c|c|c|c|}
        \hline
        & &\makecell{Метод\\Ньютона} &\makecell{Модифицир.\\метод Ньютона} &\makecell{Метод\\Марквардта} \\
        \hline
	\multirow{12}{*}{\rotatebox[origin=c]{90}{$\varepsilon = 0.01$}}&\textbf{Начальная точка} &\multicolumn{3}{c|}{\textbf{(-6.00, 2.00)}}\\
	\cline{2-5}
	&Точка минимума &(1.00, 1.00) &(1.00, 1.00) &(1.00, 1.00) \\ 
	\cline{2-5}
	&Минимум &0.00 &0.00 &0.00 \\ 
	\cline{2-5}
	&Кол-во итераций &3 &21 &18 \\ 
	\cline{2-5}
	&\makecell{Кол-во вызовов\\целевой функции} &0 &432 &28 \\ 
	\cline{2-5}
	&\makecell{Кол-во вычислений\\градиента} &4 &22 &19 \\ 
	\cline{2-5}
	&\makecell{Кол-во вычислений\\матриц Гессе} &3 &21 &18 \\ 
	\cline{2-5}
\cline{2-5}&\textbf{Начальная точка} &\multicolumn{3}{c|}{\textbf{(20.00, -30.00)}}\\
	\cline{2-5}
	&Точка минимума &(1.00, 1.00) &(1.00, 1.00) &(1.00, 1.00) \\ 
	\cline{2-5}
	&Минимум &0.00 &0.00 &0.00 \\ 
	\cline{2-5}
	&Кол-во итераций &3 &21 &24 \\ 
	\cline{2-5}
	&\makecell{Кол-во вызовов\\целевой функции} &0 &432 &38 \\ 
	\cline{2-5}
	&\makecell{Кол-во вычислений\\градиента} &4 &22 &25 \\ 
	\cline{2-5}
	&\makecell{Кол-во вычислений\\матриц Гессе} &3 &21 &24 \\ 
	\cline{2-5}
	\hline
	\multirow{12}{*}{\rotatebox[origin=c]{90}{$\varepsilon = 1e-06$}}&\textbf{Начальная точка} &\multicolumn{3}{c|}{\textbf{(-6.000000, 2.000000)}}\\
	\cline{2-5}
	&Точка минимума &(1.000000, 1.000000) &(1.000000, 1.000000) &(1.000000, 1.000000) \\ 
	\cline{2-5}
	&Минимум &0.000000 &0.000000 &0.000000 \\ 
	\cline{2-5}
	&Кол-во итераций &5 &22 &20 \\ 
	\cline{2-5}
	&\makecell{Кол-во вызовов\\целевой функции} &0 &876 &30 \\ 
	\cline{2-5}
	&\makecell{Кол-во вычислений\\градиента} &6 &23 &21 \\ 
	\cline{2-5}
	&\makecell{Кол-во вычислений\\матриц Гессе} &5 &22 &20 \\ 
	\cline{2-5}
\cline{2-5}&\textbf{Начальная точка} &\multicolumn{3}{c|}{\textbf{(20.000000, -30.000000)}}\\
	\cline{2-5}
	&Точка минимума &(1.000000, 1.000000) &(1.000000, 1.000000) &(1.000000, 1.000000) \\ 
	\cline{2-5}
	&Минимум &0.000000 &0.000000 &0.000000 \\ 
	\cline{2-5}
	&Кол-во итераций &5 &22 &25 \\ 
	\cline{2-5}
	&\makecell{Кол-во вызовов\\целевой функции} &0 &876 &39 \\ 
	\cline{2-5}
	&\makecell{Кол-во вычислений\\градиента} &6 &23 &26 \\ 
	\cline{2-5}
	&\makecell{Кол-во вычислений\\матриц Гессе} &5 &22 &25 \\ 
	\cline{2-5}
	\hline

\end{tabular}
\end{table}


            \begin{figure}[H]
	        \centering
	        \includegraphics[width=0.70\textwidth]{Классический метод Ньютона, eps 0.01, start = (-6.00, 2.00), Функция Розенброка с alpha = 10}%
	        \caption{Поиск минимума функции Розенброка с $\alpha$ = 10 при $\varepsilon = 0.01$, начальной точке (-6.0, 2.0) классическим методом Ньютона}
	        \vspace*{-1.2cm}
            \end{figure}
            
            \begin{figure}[H]
	        \centering
	        \includegraphics[width=0.70\textwidth]{Модификация метода Ньютона с наискорейшим спуском, eps 0.01, start = (-6.00, 2.00), Функция Розенброка с alpha = 10}%
	        \caption{Поиск минимума функции Розенброка с $\alpha$ = 10 при $\varepsilon = 0.01$, начальной точке (-6.0, 2.0) методом Ньютона с наискорейшим спуском}
	        \vspace*{-1.2cm}
            \end{figure}
            
            \begin{figure}[H]
	        \centering
	        \includegraphics[width=0.70\textwidth]{Метод Марквардта, eps 0.01, start = (-6.00, 2.00), Функция Розенброка с alpha = 10}%
	        \caption{Поиск минимума функции Розенброка с $\alpha$ = 10 при $\varepsilon = 0.01$, начальной точке (-6.0, 2.0) методом Марквардта}
	        \vspace*{-1.2cm}
            \end{figure}
            
            \begin{figure}[H]
	        \centering
	        \includegraphics[width=0.70\textwidth]{Классический метод Ньютона, eps 0.01, start = (20.00, -30.00), Функция Розенброка с alpha = 10}%
	        \caption{Поиск минимума функции Розенброка с $\alpha$ = 10 при $\varepsilon = 0.01$, начальной точке (20.0, -30.0) классическим методом Ньютона}
	        \vspace*{-1.2cm}
            \end{figure}
            
            \begin{figure}[H]
	        \centering
	        \includegraphics[width=0.70\textwidth]{Модификация метода Ньютона с наискорейшим спуском, eps 0.01, start = (20.00, -30.00), Функция Розенброка с alpha = 10}%
	        \caption{Поиск минимума функции Розенброка с $\alpha$ = 10 при $\varepsilon = 0.01$, начальной точке (20.0, -30.0) методом Ньютона с наискорейшим спуском}
	        \vspace*{-1.2cm}
            \end{figure}
            
            \begin{figure}[H]
	        \centering
	        \includegraphics[width=0.70\textwidth]{Метод Марквардта, eps 0.01, start = (20.00, -30.00), Функция Розенброка с alpha = 10}%
	        \caption{Поиск минимума функции Розенброка с $\alpha$ = 10 при $\varepsilon = 0.01$, начальной точке (20.0, -30.0) методом Марквардта}
	        \vspace*{-1.2cm}
            \end{figure}
            
            \begin{figure}[H]
	        \centering
	        \includegraphics[width=0.70\textwidth]{Классический метод Ньютона, eps 1e-06, start = (-6.000000, 2.000000), Функция Розенброка с alpha = 10}%
	        \caption{Поиск минимума функции Розенброка с $\alpha$ = 10 при $\varepsilon = 1e-06$, начальной точке (-6.0, 2.0) классическим методом Ньютона}
	        \vspace*{-1.2cm}
            \end{figure}
            
            \begin{figure}[H]
	        \centering
	        \includegraphics[width=0.70\textwidth]{Модификация метода Ньютона с наискорейшим спуском, eps 1e-06, start = (-6.000000, 2.000000), Функция Розенброка с alpha = 10}%
	        \caption{Поиск минимума функции Розенброка с $\alpha$ = 10 при $\varepsilon = 1e-06$, начальной точке (-6.0, 2.0) методом Ньютона с наискорейшим спуском}
	        \vspace*{-1.2cm}
            \end{figure}
            
            \begin{figure}[H]
	        \centering
	        \includegraphics[width=0.70\textwidth]{Метод Марквардта, eps 1e-06, start = (-6.000000, 2.000000), Функция Розенброка с alpha = 10}%
	        \caption{Поиск минимума функции Розенброка с $\alpha$ = 10 при $\varepsilon = 1e-06$, начальной точке (-6.0, 2.0) методом Марквардта}
	        \vspace*{-1.2cm}
            \end{figure}
            
            \begin{figure}[H]
	        \centering
	        \includegraphics[width=0.70\textwidth]{Классический метод Ньютона, eps 1e-06, start = (20.000000, -30.000000), Функция Розенброка с alpha = 10}%
	        \caption{Поиск минимума функции Розенброка с $\alpha$ = 10 при $\varepsilon = 1e-06$, начальной точке (20.0, -30.0) классическим методом Ньютона}
	        \vspace*{-1.2cm}
            \end{figure}
            
            \begin{figure}[H]
	        \centering
	        \includegraphics[width=0.70\textwidth]{Модификация метода Ньютона с наискорейшим спуском, eps 1e-06, start = (20.000000, -30.000000), Функция Розенброка с alpha = 10}%
	        \caption{Поиск минимума функции Розенброка с $\alpha$ = 10 при $\varepsilon = 1e-06$, начальной точке (20.0, -30.0) методом Ньютона с наискорейшим спуском}
	        \vspace*{-1.2cm}
            \end{figure}
            
            \begin{figure}[H]
	        \centering
	        \includegraphics[width=0.70\textwidth]{Метод Марквардта, eps 1e-06, start = (20.000000, -30.000000), Функция Розенброка с alpha = 10}%
	        \caption{Поиск минимума функции Розенброка с $\alpha$ = 10 при $\varepsilon = 1e-06$, начальной точке (20.0, -30.0) методом Марквардта}
	        \vspace*{-1.2cm}
            \end{figure}
            