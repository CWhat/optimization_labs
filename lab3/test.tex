\subsection{Квадратичная функция}

\begin{table}[H]
        \centering
        \vspace*{-1.5em}
        \caption{Результаты работы алгоритмов\\для квадратичной функции}
        \footnotesize
        \begin{tabular}{|c|c|c|c|c|}
        \hline
        & &\makecell{Метод\\сопряженных\\градиентов} &\makecell{Метод\\Флетчера --- Ривса} &\makecell{Метод\\Полака --- Рибера} \\
        \hline
	\multirow{10}{*}{\rotatebox[origin=c]{90}{$\varepsilon = 0.01$}}&\textbf{Начальная точка} &\multicolumn{3}{c|}{\textbf{(20.00, -30.00)}}\\
	\cline{2-5}
	&Точка минимума &(2.24, 0.00) &(2.24, 0.00) &(2.24, 0.00) \\ 
	\cline{2-5}
	&Минимум &-66.00 &-66.00 &-66.00 \\ 
	\cline{2-5}
	&Кол-во итераций &4 &4 &6 \\ 
	\cline{2-5}
	&\makecell{Кол-во вызовов\\целевой функции} &68 &68 &102 \\ 
	\cline{2-5}
	&\makecell{Кол-во вычислений\\градиента} &5 &5 &7 \\ 
	\cline{2-5}
\cline{2-5}&\textbf{Начальная точка} &\multicolumn{3}{c|}{\textbf{(-6.00, 2.00)}}\\
	\cline{2-5}
	&Точка минимума &(2.24, -0.00) &(2.24, -0.00) &(2.24, -0.00) \\ 
	\cline{2-5}
	&Минимум &-66.00 &-66.00 &-66.00 \\ 
	\cline{2-5}
	&Кол-во итераций &5 &5 &5 \\ 
	\cline{2-5}
	&\makecell{Кол-во вызовов\\целевой функции} &85 &85 &85 \\ 
	\cline{2-5}
	&\makecell{Кол-во вычислений\\градиента} &6 &6 &6 \\ 
	\cline{2-5}
	\hline
	\multirow{10}{*}{\rotatebox[origin=c]{90}{$\varepsilon = 1e-06$}}&\textbf{Начальная точка} &\multicolumn{3}{c|}{\textbf{(20.000000, -30.000000)}}\\
	\cline{2-5}
	&Точка минимума &(2.236068, -0.000000) &(2.236068, -0.000000) &(2.236068, 0.000000) \\ 
	\cline{2-5}
	&Минимум &-66.000000 &-66.000000 &-66.000000 \\ 
	\cline{2-5}
	&Кол-во итераций &5 &5 &9 \\ 
	\cline{2-5}
	&\makecell{Кол-во вызовов\\целевой функции} &180 &180 &324 \\ 
	\cline{2-5}
	&\makecell{Кол-во вычислений\\градиента} &6 &6 &10 \\ 
	\cline{2-5}
\cline{2-5}&\textbf{Начальная точка} &\multicolumn{3}{c|}{\textbf{(-6.000000, 2.000000)}}\\
	\cline{2-5}
	&Точка минимума &(2.236068, 0.000000) &(2.236068, 0.000000) &(2.236068, 0.000000) \\ 
	\cline{2-5}
	&Минимум &-66.000000 &-66.000000 &-66.000000 \\ 
	\cline{2-5}
	&Кол-во итераций &4 &4 &4 \\ 
	\cline{2-5}
	&\makecell{Кол-во вызовов\\целевой функции} &144 &144 &144 \\ 
	\cline{2-5}
	&\makecell{Кол-во вычислений\\градиента} &5 &5 &5 \\ 
	\cline{2-5}
	\hline

\end{tabular}
\end{table}


            \begin{figure}[H]
	        \centering
	        \includegraphics[width=0.70\textwidth]{Метод сопряженных градиентов, eps 0.01, start = (20.00, -30.00), Квадратичная функция}%
	        \caption{Поиск минимума квадратичной функции при $\varepsilon = 0.01$, начальной точке (20.0, -30.0) методом сопряженных градиентов}
	        \vspace*{-1.2cm}
            \end{figure}
            
            \begin{figure}[H]
	        \centering
	        \includegraphics[width=0.70\textwidth]{Метод Флетчера-Ривса, eps 0.01, start = (20.00, -30.00), Квадратичная функция}%
	        \caption{Поиск минимума квадратичной функции при $\varepsilon = 0.01$, начальной точке (20.0, -30.0) методом Флетчера --- Ривса}
	        \vspace*{-1.2cm}
            \end{figure}
            
            \begin{figure}[H]
	        \centering
	        \includegraphics[width=0.70\textwidth]{Метод Полака-Рибера, eps 0.01, start = (20.00, -30.00), Квадратичная функция}%
	        \caption{Поиск минимума квадратичной функции при $\varepsilon = 0.01$, начальной точке (20.0, -30.0) методом Полака --- Рибера}
	        \vspace*{-1.2cm}
            \end{figure}
            
            \begin{figure}[H]
	        \centering
	        \includegraphics[width=0.70\textwidth]{Метод сопряженных градиентов, eps 0.01, start = (-6.00, 2.00), Квадратичная функция}%
	        \caption{Поиск минимума квадратичной функции при $\varepsilon = 0.01$, начальной точке (-6.0, 2.0) методом сопряженных градиентов}
	        \vspace*{-1.2cm}
            \end{figure}
            
            \begin{figure}[H]
	        \centering
	        \includegraphics[width=0.70\textwidth]{Метод Флетчера-Ривса, eps 0.01, start = (-6.00, 2.00), Квадратичная функция}%
	        \caption{Поиск минимума квадратичной функции при $\varepsilon = 0.01$, начальной точке (-6.0, 2.0) методом Флетчера --- Ривса}
	        \vspace*{-1.2cm}
            \end{figure}
            
            \begin{figure}[H]
	        \centering
	        \includegraphics[width=0.70\textwidth]{Метод Полака-Рибера, eps 0.01, start = (-6.00, 2.00), Квадратичная функция}%
	        \caption{Поиск минимума квадратичной функции при $\varepsilon = 0.01$, начальной точке (-6.0, 2.0) методом Полака --- Рибера}
	        \vspace*{-1.2cm}
            \end{figure}
            
            \begin{figure}[H]
	        \centering
	        \includegraphics[width=0.70\textwidth]{Метод сопряженных градиентов, eps 1e-06, start = (20.000000, -30.000000), Квадратичная функция}%
	        \caption{Поиск минимума квадратичной функции при $\varepsilon = 1e-06$, начальной точке (20.0, -30.0) методом сопряженных градиентов}
	        \vspace*{-1.2cm}
            \end{figure}
            
            \begin{figure}[H]
	        \centering
	        \includegraphics[width=0.70\textwidth]{Метод Флетчера-Ривса, eps 1e-06, start = (20.000000, -30.000000), Квадратичная функция}%
	        \caption{Поиск минимума квадратичной функции при $\varepsilon = 1e-06$, начальной точке (20.0, -30.0) методом Флетчера --- Ривса}
	        \vspace*{-1.2cm}
            \end{figure}
            
            \begin{figure}[H]
	        \centering
	        \includegraphics[width=0.70\textwidth]{Метод Полака-Рибера, eps 1e-06, start = (20.000000, -30.000000), Квадратичная функция}%
	        \caption{Поиск минимума квадратичной функции при $\varepsilon = 1e-06$, начальной точке (20.0, -30.0) методом Полака --- Рибера}
	        \vspace*{-1.2cm}
            \end{figure}
            
            \begin{figure}[H]
	        \centering
	        \includegraphics[width=0.70\textwidth]{Метод сопряженных градиентов, eps 1e-06, start = (-6.000000, 2.000000), Квадратичная функция}%
	        \caption{Поиск минимума квадратичной функции при $\varepsilon = 1e-06$, начальной точке (-6.0, 2.0) методом сопряженных градиентов}
	        \vspace*{-1.2cm}
            \end{figure}
            
            \begin{figure}[H]
	        \centering
	        \includegraphics[width=0.70\textwidth]{Метод Флетчера-Ривса, eps 1e-06, start = (-6.000000, 2.000000), Квадратичная функция}%
	        \caption{Поиск минимума квадратичной функции при $\varepsilon = 1e-06$, начальной точке (-6.0, 2.0) методом Флетчера --- Ривса}
	        \vspace*{-1.2cm}
            \end{figure}
            
            \begin{figure}[H]
	        \centering
	        \includegraphics[width=0.70\textwidth]{Метод Полака-Рибера, eps 1e-06, start = (-6.000000, 2.000000), Квадратичная функция}%
	        \caption{Поиск минимума квадратичной функции при $\varepsilon = 1e-06$, начальной точке (-6.0, 2.0) методом Полака --- Рибера}
	        \vspace*{-1.2cm}
            \end{figure}
            \subsection{Функция Розенброка с $\alpha$ = 1}

\begin{table}[H]
        \centering
        \vspace*{-1.5em}
        \caption{Результаты работы алгоритмов\\для функции Розенброка с $\alpha$ = 1}
        \footnotesize
        \begin{tabular}{|c|c|c|c|c|}
        \hline
        & &\makecell{Метод\\сопряженных\\градиентов} &\makecell{Метод\\Флетчера --- Ривса} &\makecell{Метод\\Полака --- Рибера} \\
        \hline
	\multirow{10}{*}{\rotatebox[origin=c]{90}{$\varepsilon = 0.01$}}&\textbf{Начальная точка} &\multicolumn{3}{c|}{\textbf{(20.00, -30.00)}}\\
	\cline{2-5}
	&Точка минимума &(1.00, 1.00) &(0.99, 0.99) &(0.99, 0.98) \\ 
	\cline{2-5}
	&Минимум &0.00 &0.00 &0.00 \\ 
	\cline{2-5}
	&Кол-во итераций &7 &9 &10 \\ 
	\cline{2-5}
	&\makecell{Кол-во вызовов\\целевой функции} &119 &153 &170 \\ 
	\cline{2-5}
	&\makecell{Кол-во вычислений\\градиента} &8 &10 &11 \\ 
	\cline{2-5}
\cline{2-5}&\textbf{Начальная точка} &\multicolumn{3}{c|}{\textbf{(-6.00, 2.00)}}\\
	\cline{2-5}
	&Точка минимума &(1.00, 1.01) &(1.00, 0.99) &(1.00, 1.01) \\ 
	\cline{2-5}
	&Минимум &0.00 &0.00 &0.00 \\ 
	\cline{2-5}
	&Кол-во итераций &10 &12 &7 \\ 
	\cline{2-5}
	&\makecell{Кол-во вызовов\\целевой функции} &170 &204 &119 \\ 
	\cline{2-5}
	&\makecell{Кол-во вычислений\\градиента} &11 &13 &8 \\ 
	\cline{2-5}
	\hline
	\multirow{10}{*}{\rotatebox[origin=c]{90}{$\varepsilon = 1e-06$}}&\textbf{Начальная точка} &\multicolumn{3}{c|}{\textbf{(20.000000, -30.000000)}}\\
	\cline{2-5}
	&Точка минимума &(1.000000, 0.999999) &(1.000000, 1.000001) &(1.000000, 1.000000) \\ 
	\cline{2-5}
	&Минимум &0.000000 &0.000000 &0.000000 \\ 
	\cline{2-5}
	&Кол-во итераций &25 &19 &16 \\ 
	\cline{2-5}
	&\makecell{Кол-во вызовов\\целевой функции} &900 &684 &576 \\ 
	\cline{2-5}
	&\makecell{Кол-во вычислений\\градиента} &26 &20 &17 \\ 
	\cline{2-5}
\cline{2-5}&\textbf{Начальная точка} &\multicolumn{3}{c|}{\textbf{(-6.000000, 2.000000)}}\\
	\cline{2-5}
	&Точка минимума &(1.000000, 1.000001) &(1.000000, 1.000001) &(1.000000, 1.000000) \\ 
	\cline{2-5}
	&Минимум &0.000000 &0.000000 &0.000000 \\ 
	\cline{2-5}
	&Кол-во итераций &22 &38 &16 \\ 
	\cline{2-5}
	&\makecell{Кол-во вызовов\\целевой функции} &792 &1368 &576 \\ 
	\cline{2-5}
	&\makecell{Кол-во вычислений\\градиента} &23 &39 &17 \\ 
	\cline{2-5}
	\hline

\end{tabular}
\end{table}


            \begin{figure}[H]
	        \centering
	        \includegraphics[width=0.70\textwidth]{Метод сопряженных градиентов, eps 0.01, start = (20.00, -30.00), Функция Розенброка с alpha = 1}%
	        \caption{Поиск минимума функции Розенброка с $\alpha$ = 1 при $\varepsilon = 0.01$, начальной точке (20.0, -30.0) методом сопряженных градиентов}
	        \vspace*{-1.2cm}
            \end{figure}
            
            \begin{figure}[H]
	        \centering
	        \includegraphics[width=0.70\textwidth]{Метод Флетчера-Ривса, eps 0.01, start = (20.00, -30.00), Функция Розенброка с alpha = 1}%
	        \caption{Поиск минимума функции Розенброка с $\alpha$ = 1 при $\varepsilon = 0.01$, начальной точке (20.0, -30.0) методом Флетчера --- Ривса}
	        \vspace*{-1.2cm}
            \end{figure}
            
            \begin{figure}[H]
	        \centering
	        \includegraphics[width=0.70\textwidth]{Метод Полака-Рибера, eps 0.01, start = (20.00, -30.00), Функция Розенброка с alpha = 1}%
	        \caption{Поиск минимума функции Розенброка с $\alpha$ = 1 при $\varepsilon = 0.01$, начальной точке (20.0, -30.0) методом Полака --- Рибера}
	        \vspace*{-1.2cm}
            \end{figure}
            
            \begin{figure}[H]
	        \centering
	        \includegraphics[width=0.70\textwidth]{Метод сопряженных градиентов, eps 0.01, start = (-6.00, 2.00), Функция Розенброка с alpha = 1}%
	        \caption{Поиск минимума функции Розенброка с $\alpha$ = 1 при $\varepsilon = 0.01$, начальной точке (-6.0, 2.0) методом сопряженных градиентов}
	        \vspace*{-1.2cm}
            \end{figure}
            
            \begin{figure}[H]
	        \centering
	        \includegraphics[width=0.70\textwidth]{Метод Флетчера-Ривса, eps 0.01, start = (-6.00, 2.00), Функция Розенброка с alpha = 1}%
	        \caption{Поиск минимума функции Розенброка с $\alpha$ = 1 при $\varepsilon = 0.01$, начальной точке (-6.0, 2.0) методом Флетчера --- Ривса}
	        \vspace*{-1.2cm}
            \end{figure}
            
            \begin{figure}[H]
	        \centering
	        \includegraphics[width=0.70\textwidth]{Метод Полака-Рибера, eps 0.01, start = (-6.00, 2.00), Функция Розенброка с alpha = 1}%
	        \caption{Поиск минимума функции Розенброка с $\alpha$ = 1 при $\varepsilon = 0.01$, начальной точке (-6.0, 2.0) методом Полака --- Рибера}
	        \vspace*{-1.2cm}
            \end{figure}
            
            \begin{figure}[H]
	        \centering
	        \includegraphics[width=0.70\textwidth]{Метод сопряженных градиентов, eps 1e-06, start = (20.000000, -30.000000), Функция Розенброка с alpha = 1}%
	        \caption{Поиск минимума функции Розенброка с $\alpha$ = 1 при $\varepsilon = 1e-06$, начальной точке (20.0, -30.0) методом сопряженных градиентов}
	        \vspace*{-1.2cm}
            \end{figure}
            
            \begin{figure}[H]
	        \centering
	        \includegraphics[width=0.70\textwidth]{Метод Флетчера-Ривса, eps 1e-06, start = (20.000000, -30.000000), Функция Розенброка с alpha = 1}%
	        \caption{Поиск минимума функции Розенброка с $\alpha$ = 1 при $\varepsilon = 1e-06$, начальной точке (20.0, -30.0) методом Флетчера --- Ривса}
	        \vspace*{-1.2cm}
            \end{figure}
            
            \begin{figure}[H]
	        \centering
	        \includegraphics[width=0.70\textwidth]{Метод Полака-Рибера, eps 1e-06, start = (20.000000, -30.000000), Функция Розенброка с alpha = 1}%
	        \caption{Поиск минимума функции Розенброка с $\alpha$ = 1 при $\varepsilon = 1e-06$, начальной точке (20.0, -30.0) методом Полака --- Рибера}
	        \vspace*{-1.2cm}
            \end{figure}
            
            \begin{figure}[H]
	        \centering
	        \includegraphics[width=0.70\textwidth]{Метод сопряженных градиентов, eps 1e-06, start = (-6.000000, 2.000000), Функция Розенброка с alpha = 1}%
	        \caption{Поиск минимума функции Розенброка с $\alpha$ = 1 при $\varepsilon = 1e-06$, начальной точке (-6.0, 2.0) методом сопряженных градиентов}
	        \vspace*{-1.2cm}
            \end{figure}
            
            \begin{figure}[H]
	        \centering
	        \includegraphics[width=0.70\textwidth]{Метод Флетчера-Ривса, eps 1e-06, start = (-6.000000, 2.000000), Функция Розенброка с alpha = 1}%
	        \caption{Поиск минимума функции Розенброка с $\alpha$ = 1 при $\varepsilon = 1e-06$, начальной точке (-6.0, 2.0) методом Флетчера --- Ривса}
	        \vspace*{-1.2cm}
            \end{figure}
            
            \begin{figure}[H]
	        \centering
	        \includegraphics[width=0.70\textwidth]{Метод Полака-Рибера, eps 1e-06, start = (-6.000000, 2.000000), Функция Розенброка с alpha = 1}%
	        \caption{Поиск минимума функции Розенброка с $\alpha$ = 1 при $\varepsilon = 1e-06$, начальной точке (-6.0, 2.0) методом Полака --- Рибера}
	        \vspace*{-1.2cm}
            \end{figure}
            \subsection{Функция Розенброка с $\alpha$ = 10}

\begin{table}[H]
        \centering
        \vspace*{-1.5em}
        \caption{Результаты работы алгоритмов\\для функции Розенброка с $\alpha$ = 10}
        \footnotesize
        \begin{tabular}{|c|c|c|c|c|}
        \hline
        & &\makecell{Метод\\сопряженных\\градиентов} &\makecell{Метод\\Флетчера --- Ривса} &\makecell{Метод\\Полака --- Рибера} \\
        \hline
	\multirow{10}{*}{\rotatebox[origin=c]{90}{$\varepsilon = 0.01$}}&\textbf{Начальная точка} &\multicolumn{3}{c|}{\textbf{(20.00, -30.00)}}\\
	\cline{2-5}
	&Точка минимума &(1.00, 0.99) &(1.00, 1.00) &(1.00, 1.00) \\ 
	\cline{2-5}
	&Минимум &0.00 &0.00 &0.00 \\ 
	\cline{2-5}
	&Кол-во итераций &18 &13 &8 \\ 
	\cline{2-5}
	&\makecell{Кол-во вызовов\\целевой функции} &306 &221 &136 \\ 
	\cline{2-5}
	&\makecell{Кол-во вычислений\\градиента} &19 &14 &9 \\ 
	\cline{2-5}
\cline{2-5}&\textbf{Начальная точка} &\multicolumn{3}{c|}{\textbf{(-6.00, 2.00)}}\\
	\cline{2-5}
	&Точка минимума &(1.01, 1.01) &(0.99, 0.99) &(1.00, 1.00) \\ 
	\cline{2-5}
	&Минимум &0.00 &0.00 &0.00 \\ 
	\cline{2-5}
	&Кол-во итераций &18 &25 &16 \\ 
	\cline{2-5}
	&\makecell{Кол-во вызовов\\целевой функции} &306 &425 &272 \\ 
	\cline{2-5}
	&\makecell{Кол-во вычислений\\градиента} &19 &26 &17 \\ 
	\cline{2-5}
	\hline
	\multirow{10}{*}{\rotatebox[origin=c]{90}{$\varepsilon = 1e-06$}}&\textbf{Начальная точка} &\multicolumn{3}{c|}{\textbf{(20.000000, -30.000000)}}\\
	\cline{2-5}
	&Точка минимума &(1.000000, 1.000000) &(1.000000, 1.000000) &(1.000000, 1.000000) \\ 
	\cline{2-5}
	&Минимум &0.000000 &0.000000 &0.000000 \\ 
	\cline{2-5}
	&Кол-во итераций &30 &23 &25 \\ 
	\cline{2-5}
	&\makecell{Кол-во вызовов\\целевой функции} &1080 &828 &900 \\ 
	\cline{2-5}
	&\makecell{Кол-во вычислений\\градиента} &31 &24 &26 \\ 
	\cline{2-5}
\cline{2-5}&\textbf{Начальная точка} &\multicolumn{3}{c|}{\textbf{(-6.000000, 2.000000)}}\\
	\cline{2-5}
	&Точка минимума &(1.000000, 1.000000) &(1.000000, 1.000000) &(1.000000, 1.000000) \\ 
	\cline{2-5}
	&Минимум &0.000000 &0.000000 &0.000000 \\ 
	\cline{2-5}
	&Кол-во итераций &30 &43 &15 \\ 
	\cline{2-5}
	&\makecell{Кол-во вызовов\\целевой функции} &1080 &1548 &540 \\ 
	\cline{2-5}
	&\makecell{Кол-во вычислений\\градиента} &31 &44 &16 \\ 
	\cline{2-5}
	\hline

\end{tabular}
\end{table}


            \begin{figure}[H]
	        \centering
	        \includegraphics[width=0.70\textwidth]{Метод сопряженных градиентов, eps 0.01, start = (20.00, -30.00), Функция Розенброка с alpha = 10}%
	        \caption{Поиск минимума функции Розенброка с $\alpha$ = 10 при $\varepsilon = 0.01$, начальной точке (20.0, -30.0) методом сопряженных градиентов}
	        \vspace*{-1.2cm}
            \end{figure}
            
            \begin{figure}[H]
	        \centering
	        \includegraphics[width=0.70\textwidth]{Метод Флетчера-Ривса, eps 0.01, start = (20.00, -30.00), Функция Розенброка с alpha = 10}%
	        \caption{Поиск минимума функции Розенброка с $\alpha$ = 10 при $\varepsilon = 0.01$, начальной точке (20.0, -30.0) методом Флетчера --- Ривса}
	        \vspace*{-1.2cm}
            \end{figure}
            
            \begin{figure}[H]
	        \centering
	        \includegraphics[width=0.70\textwidth]{Метод Полака-Рибера, eps 0.01, start = (20.00, -30.00), Функция Розенброка с alpha = 10}%
	        \caption{Поиск минимума функции Розенброка с $\alpha$ = 10 при $\varepsilon = 0.01$, начальной точке (20.0, -30.0) методом Полака --- Рибера}
	        \vspace*{-1.2cm}
            \end{figure}
            
            \begin{figure}[H]
	        \centering
	        \includegraphics[width=0.70\textwidth]{Метод сопряженных градиентов, eps 0.01, start = (-6.00, 2.00), Функция Розенброка с alpha = 10}%
	        \caption{Поиск минимума функции Розенброка с $\alpha$ = 10 при $\varepsilon = 0.01$, начальной точке (-6.0, 2.0) методом сопряженных градиентов}
	        \vspace*{-1.2cm}
            \end{figure}
            
            \begin{figure}[H]
	        \centering
	        \includegraphics[width=0.70\textwidth]{Метод Флетчера-Ривса, eps 0.01, start = (-6.00, 2.00), Функция Розенброка с alpha = 10}%
	        \caption{Поиск минимума функции Розенброка с $\alpha$ = 10 при $\varepsilon = 0.01$, начальной точке (-6.0, 2.0) методом Флетчера --- Ривса}
	        \vspace*{-1.2cm}
            \end{figure}
            
            \begin{figure}[H]
	        \centering
	        \includegraphics[width=0.70\textwidth]{Метод Полака-Рибера, eps 0.01, start = (-6.00, 2.00), Функция Розенброка с alpha = 10}%
	        \caption{Поиск минимума функции Розенброка с $\alpha$ = 10 при $\varepsilon = 0.01$, начальной точке (-6.0, 2.0) методом Полака --- Рибера}
	        \vspace*{-1.2cm}
            \end{figure}
            
            \begin{figure}[H]
	        \centering
	        \includegraphics[width=0.70\textwidth]{Метод сопряженных градиентов, eps 1e-06, start = (20.000000, -30.000000), Функция Розенброка с alpha = 10}%
	        \caption{Поиск минимума функции Розенброка с $\alpha$ = 10 при $\varepsilon = 1e-06$, начальной точке (20.0, -30.0) методом сопряженных градиентов}
	        \vspace*{-1.2cm}
            \end{figure}
            
            \begin{figure}[H]
	        \centering
	        \includegraphics[width=0.70\textwidth]{Метод Флетчера-Ривса, eps 1e-06, start = (20.000000, -30.000000), Функция Розенброка с alpha = 10}%
	        \caption{Поиск минимума функции Розенброка с $\alpha$ = 10 при $\varepsilon = 1e-06$, начальной точке (20.0, -30.0) методом Флетчера --- Ривса}
	        \vspace*{-1.2cm}
            \end{figure}
            
            \begin{figure}[H]
	        \centering
	        \includegraphics[width=0.70\textwidth]{Метод Полака-Рибера, eps 1e-06, start = (20.000000, -30.000000), Функция Розенброка с alpha = 10}%
	        \caption{Поиск минимума функции Розенброка с $\alpha$ = 10 при $\varepsilon = 1e-06$, начальной точке (20.0, -30.0) методом Полака --- Рибера}
	        \vspace*{-1.2cm}
            \end{figure}
            
            \begin{figure}[H]
	        \centering
	        \includegraphics[width=0.70\textwidth]{Метод сопряженных градиентов, eps 1e-06, start = (-6.000000, 2.000000), Функция Розенброка с alpha = 10}%
	        \caption{Поиск минимума функции Розенброка с $\alpha$ = 10 при $\varepsilon = 1e-06$, начальной точке (-6.0, 2.0) методом сопряженных градиентов}
	        \vspace*{-1.2cm}
            \end{figure}
            
            \begin{figure}[H]
	        \centering
	        \includegraphics[width=0.70\textwidth]{Метод Флетчера-Ривса, eps 1e-06, start = (-6.000000, 2.000000), Функция Розенброка с alpha = 10}%
	        \caption{Поиск минимума функции Розенброка с $\alpha$ = 10 при $\varepsilon = 1e-06$, начальной точке (-6.0, 2.0) методом Флетчера --- Ривса}
	        \vspace*{-1.2cm}
            \end{figure}
            
            \begin{figure}[H]
	        \centering
	        \includegraphics[width=0.70\textwidth]{Метод Полака-Рибера, eps 1e-06, start = (-6.000000, 2.000000), Функция Розенброка с alpha = 10}%
	        \caption{Поиск минимума функции Розенброка с $\alpha$ = 10 при $\varepsilon = 1e-06$, начальной точке (-6.0, 2.0) методом Полака --- Рибера}
	        \vspace*{-1.2cm}
            \end{figure}
            