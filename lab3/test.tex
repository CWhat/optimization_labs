\subsection{Квадратичная функция}

\begin{table}[H]
        \centering
        \vspace*{-1.5em}
        \caption{Результаты работы алгоритмов\\для квадратичной функции}
        \footnotesize
        \begin{tabular}{|c|c|c|c|c|}
        \hline
        & &\makecell{Метод\\сопряженных\\градиентов} &\makecell{Метод\\Флетчера --- Ривса} &\makecell{Метод\\Полака --- Рибера} \\
        \hline
	\multirow{12}{*}{\rotatebox[origin=c]{90}{$\varepsilon = 0.01$}}&\textbf{Начальная точка} &\multicolumn{3}{c|}{\textbf{(3.00, 20.00)}}\\
	\cline{2-5}
	&Точка минимума &(2.24, 0.00) &(2.24, 0.00) &(2.24, -0.00) \\ 
	\cline{2-5}
	&Минимум &-66.00 &-66.00 &-66.00 \\ 
	\cline{2-5}
	&Кол-во итераций &15 &5 &6 \\ 
	\cline{2-5}
	&\makecell{Кол-во вызовов\\целевой функции} &254 &82 &143 \\ 
	\cline{2-5}
	&\makecell{Кол-во вычислений\\градиента} &16 &6 &7 \\ 
	\cline{2-5}
	&\makecell{Кол-во вычислений\\матриц Гессе} &15 &0 &0 \\ 
	\cline{2-5}
\cline{2-5}&\textbf{Начальная точка} &\multicolumn{3}{c|}{\textbf{(-5.00, 6.00)}}\\
	\cline{2-5}
	&Точка минимума &(2.24, 0.00) &(2.24, 0.00) &(2.24, 0.00) \\ 
	\cline{2-5}
	&Минимум &-66.00 &-66.00 &-66.00 \\ 
	\cline{2-5}
	&Кол-во итераций &47 &4 &4 \\ 
	\cline{2-5}
	&\makecell{Кол-во вызовов\\целевой функции} &967 &66 &66 \\ 
	\cline{2-5}
	&\makecell{Кол-во вычислений\\градиента} &48 &5 &5 \\ 
	\cline{2-5}
	&\makecell{Кол-во вычислений\\матриц Гессе} &47 &0 &0 \\ 
	\cline{2-5}
	\hline
	\multirow{12}{*}{\rotatebox[origin=c]{90}{$\varepsilon = 1e-06$}}&\textbf{Начальная точка} &\multicolumn{3}{c|}{\textbf{(3.000000, 20.000000)}}\\
	\cline{2-5}
	&Точка минимума &(2.236068, -0.000000) &(2.236068, -0.000000) &(2.236068, -0.000000) \\ 
	\cline{2-5}
	&Минимум &-66.000000 &-66.000000 &-66.000000 \\ 
	\cline{2-5}
	&Кол-во итераций &29 &5 &6 \\ 
	\cline{2-5}
	&\makecell{Кол-во вызовов\\целевой функции} &1038 &174 &211 \\ 
	\cline{2-5}
	&\makecell{Кол-во вычислений\\градиента} &30 &6 &7 \\ 
	\cline{2-5}
	&\makecell{Кол-во вычислений\\матриц Гессе} &29 &0 &0 \\ 
	\cline{2-5}
\cline{2-5}&\textbf{Начальная точка} &\multicolumn{3}{c|}{\textbf{(-5.000000, 6.000000)}}\\
	\cline{2-5}
	&Точка минимума &(2.236068, -0.000000) &(2.236068, 0.000000) &(2.236068, 0.000000) \\ 
	\cline{2-5}
	&Минимум &-66.000000 &-66.000000 &-66.000000 \\ 
	\cline{2-5}
	&Кол-во итераций &67 &4 &5 \\ 
	\cline{2-5}
	&\makecell{Кол-во вызовов\\целевой функции} &2594 &142 &172 \\ 
	\cline{2-5}
	&\makecell{Кол-во вычислений\\градиента} &68 &5 &6 \\ 
	\cline{2-5}
	&\makecell{Кол-во вычислений\\матриц Гессе} &67 &0 &0 \\ 
	\cline{2-5}
	\hline

\end{tabular}
\end{table}


            \begin{figure}[H]
	        \centering
	        \includegraphics[width=0.70\textwidth]{Метод сопряженных градиентов, eps 0.01, start = (3.00, 20.00), Квадратичная функция}%
	        \caption{Поиск минимума квадратичной функции при $\varepsilon = 0.01$, начальной точке (3.0, 20.0) методом сопряженных градиентов}
	        \vspace*{-1.2cm}
            \end{figure}
            
            \begin{figure}[H]
	        \centering
	        \includegraphics[width=0.70\textwidth]{Метод Флетчера-Ривса, eps 0.01, start = (3.00, 20.00), Квадратичная функция}%
	        \caption{Поиск минимума квадратичной функции при $\varepsilon = 0.01$, начальной точке (3.0, 20.0) методом Флетчера --- Ривса}
	        \vspace*{-1.2cm}
            \end{figure}
            
            \begin{figure}[H]
	        \centering
	        \includegraphics[width=0.70\textwidth]{Метод Полака-Рибера, eps 0.01, start = (3.00, 20.00), Квадратичная функция}%
	        \caption{Поиск минимума квадратичной функции при $\varepsilon = 0.01$, начальной точке (3.0, 20.0) методом Полака --- Рибера}
	        \vspace*{-1.2cm}
            \end{figure}
            
            \begin{figure}[H]
	        \centering
	        \includegraphics[width=0.70\textwidth]{Метод сопряженных градиентов, eps 0.01, start = (-5.00, 6.00), Квадратичная функция}%
	        \caption{Поиск минимума квадратичной функции при $\varepsilon = 0.01$, начальной точке (-5.0, 6.0) методом сопряженных градиентов}
	        \vspace*{-1.2cm}
            \end{figure}
            
            \begin{figure}[H]
	        \centering
	        \includegraphics[width=0.70\textwidth]{Метод Флетчера-Ривса, eps 0.01, start = (-5.00, 6.00), Квадратичная функция}%
	        \caption{Поиск минимума квадратичной функции при $\varepsilon = 0.01$, начальной точке (-5.0, 6.0) методом Флетчера --- Ривса}
	        \vspace*{-1.2cm}
            \end{figure}
            
            \begin{figure}[H]
	        \centering
	        \includegraphics[width=0.70\textwidth]{Метод Полака-Рибера, eps 0.01, start = (-5.00, 6.00), Квадратичная функция}%
	        \caption{Поиск минимума квадратичной функции при $\varepsilon = 0.01$, начальной точке (-5.0, 6.0) методом Полака --- Рибера}
	        \vspace*{-1.2cm}
            \end{figure}
            
            \begin{figure}[H]
	        \centering
	        \includegraphics[width=0.70\textwidth]{Метод сопряженных градиентов, eps 1e-06, start = (3.000000, 20.000000), Квадратичная функция}%
	        \caption{Поиск минимума квадратичной функции при $\varepsilon = 1e-06$, начальной точке (3.0, 20.0) методом сопряженных градиентов}
	        \vspace*{-1.2cm}
            \end{figure}
            
            \begin{figure}[H]
	        \centering
	        \includegraphics[width=0.70\textwidth]{Метод Флетчера-Ривса, eps 1e-06, start = (3.000000, 20.000000), Квадратичная функция}%
	        \caption{Поиск минимума квадратичной функции при $\varepsilon = 1e-06$, начальной точке (3.0, 20.0) методом Флетчера --- Ривса}
	        \vspace*{-1.2cm}
            \end{figure}
            
            \begin{figure}[H]
	        \centering
	        \includegraphics[width=0.70\textwidth]{Метод Полака-Рибера, eps 1e-06, start = (3.000000, 20.000000), Квадратичная функция}%
	        \caption{Поиск минимума квадратичной функции при $\varepsilon = 1e-06$, начальной точке (3.0, 20.0) методом Полака --- Рибера}
	        \vspace*{-1.2cm}
            \end{figure}
            
            \begin{figure}[H]
	        \centering
	        \includegraphics[width=0.70\textwidth]{Метод сопряженных градиентов, eps 1e-06, start = (-5.000000, 6.000000), Квадратичная функция}%
	        \caption{Поиск минимума квадратичной функции при $\varepsilon = 1e-06$, начальной точке (-5.0, 6.0) методом сопряженных градиентов}
	        \vspace*{-1.2cm}
            \end{figure}
            
            \begin{figure}[H]
	        \centering
	        \includegraphics[width=0.70\textwidth]{Метод Флетчера-Ривса, eps 1e-06, start = (-5.000000, 6.000000), Квадратичная функция}%
	        \caption{Поиск минимума квадратичной функции при $\varepsilon = 1e-06$, начальной точке (-5.0, 6.0) методом Флетчера --- Ривса}
	        \vspace*{-1.2cm}
            \end{figure}
            
            \begin{figure}[H]
	        \centering
	        \includegraphics[width=0.70\textwidth]{Метод Полака-Рибера, eps 1e-06, start = (-5.000000, 6.000000), Квадратичная функция}%
	        \caption{Поиск минимума квадратичной функции при $\varepsilon = 1e-06$, начальной точке (-5.0, 6.0) методом Полака --- Рибера}
	        \vspace*{-1.2cm}
            \end{figure}
            \subsection{Функция Розенброка с $\alpha$ = 1}

\begin{table}[H]
        \centering
        \vspace*{-1.5em}
        \caption{Результаты работы алгоритмов\\для функции Розенброка с $\alpha$ = 1}
        \footnotesize
        \begin{tabular}{|c|c|c|c|c|}
        \hline
        & &\makecell{Метод\\сопряженных\\градиентов} &\makecell{Метод\\Флетчера --- Ривса} &\makecell{Метод\\Полака --- Рибера} \\
        \hline
	\multirow{12}{*}{\rotatebox[origin=c]{90}{$\varepsilon = 0.01$}}&\textbf{Начальная точка} &\multicolumn{3}{c|}{\textbf{(3.00, 20.00)}}\\
	\cline{2-5}
	&Точка минимума &(1.00, 1.00) &(1.01, 1.01) &(1.00, 1.00) \\ 
	\cline{2-5}
	&Минимум &0.00 &0.00 &0.00 \\ 
	\cline{2-5}
	&Кол-во итераций &16 &14 &10 \\ 
	\cline{2-5}
	&\makecell{Кол-во вызовов\\целевой функции} &312 &212 &163 \\ 
	\cline{2-5}
	&\makecell{Кол-во вычислений\\градиента} &17 &15 &11 \\ 
	\cline{2-5}
	&\makecell{Кол-во вычислений\\матриц Гессе} &16 &0 &0 \\ 
	\cline{2-5}
\cline{2-5}&\textbf{Начальная точка} &\multicolumn{3}{c|}{\textbf{(-5.00, 6.00)}}\\
	\cline{2-5}
	&Точка минимума &(1.00, 1.00) &(1.00, 0.99) &(0.99, 0.98) \\ 
	\cline{2-5}
	&Минимум &0.00 &0.00 &0.00 \\ 
	\cline{2-5}
	&Кол-во итераций &13 &16 &10 \\ 
	\cline{2-5}
	&\makecell{Кол-во вызовов\\целевой функции} &188 &237 &142 \\ 
	\cline{2-5}
	&\makecell{Кол-во вычислений\\градиента} &14 &17 &11 \\ 
	\cline{2-5}
	&\makecell{Кол-во вычислений\\матриц Гессе} &13 &0 &0 \\ 
	\cline{2-5}
	\hline
	\multirow{12}{*}{\rotatebox[origin=c]{90}{$\varepsilon = 1e-06$}}&\textbf{Начальная точка} &\multicolumn{3}{c|}{\textbf{(3.000000, 20.000000)}}\\
	\cline{2-5}
	&Точка минимума &(1.000000, 1.000000) &(1.000000, 1.000001) &(1.000000, 1.000000) \\ 
	\cline{2-5}
	&Минимум &0.000000 &0.000000 &0.000000 \\ 
	\cline{2-5}
	&Кол-во итераций &19 &84 &16 \\ 
	\cline{2-5}
	&\makecell{Кол-во вызовов\\целевой функции} &722 &3032 &543 \\ 
	\cline{2-5}
	&\makecell{Кол-во вычислений\\градиента} &20 &85 &17 \\ 
	\cline{2-5}
	&\makecell{Кол-во вычислений\\матриц Гессе} &19 &0 &0 \\ 
	\cline{2-5}
\cline{2-5}&\textbf{Начальная точка} &\multicolumn{3}{c|}{\textbf{(-5.000000, 6.000000)}}\\
	\cline{2-5}
	&Точка минимума &(1.000000, 1.000000) &(0.999999, 0.999999) &(1.000000, 1.000000) \\ 
	\cline{2-5}
	&Минимум &0.000000 &0.000000 &0.000000 \\ 
	\cline{2-5}
	&Кол-во итераций &19 &52 &10 \\ 
	\cline{2-5}
	&\makecell{Кол-во вызовов\\целевой функции} &631 &1783 &337 \\ 
	\cline{2-5}
	&\makecell{Кол-во вычислений\\градиента} &20 &53 &11 \\ 
	\cline{2-5}
	&\makecell{Кол-во вычислений\\матриц Гессе} &19 &0 &0 \\ 
	\cline{2-5}
	\hline

\end{tabular}
\end{table}


            \begin{figure}[H]
	        \centering
	        \includegraphics[width=0.70\textwidth]{Метод сопряженных градиентов, eps 0.01, start = (3.00, 20.00), Функция Розенброка с alpha = 1}%
	        \caption{Поиск минимума функции Розенброка с $\alpha$ = 1 при $\varepsilon = 0.01$, начальной точке (3.0, 20.0) методом сопряженных градиентов}
	        \vspace*{-1.2cm}
            \end{figure}
            
            \begin{figure}[H]
	        \centering
	        \includegraphics[width=0.70\textwidth]{Метод Флетчера-Ривса, eps 0.01, start = (3.00, 20.00), Функция Розенброка с alpha = 1}%
	        \caption{Поиск минимума функции Розенброка с $\alpha$ = 1 при $\varepsilon = 0.01$, начальной точке (3.0, 20.0) методом Флетчера --- Ривса}
	        \vspace*{-1.2cm}
            \end{figure}
            
            \begin{figure}[H]
	        \centering
	        \includegraphics[width=0.70\textwidth]{Метод Полака-Рибера, eps 0.01, start = (3.00, 20.00), Функция Розенброка с alpha = 1}%
	        \caption{Поиск минимума функции Розенброка с $\alpha$ = 1 при $\varepsilon = 0.01$, начальной точке (3.0, 20.0) методом Полака --- Рибера}
	        \vspace*{-1.2cm}
            \end{figure}
            
            \begin{figure}[H]
	        \centering
	        \includegraphics[width=0.70\textwidth]{Метод сопряженных градиентов, eps 0.01, start = (-5.00, 6.00), Функция Розенброка с alpha = 1}%
	        \caption{Поиск минимума функции Розенброка с $\alpha$ = 1 при $\varepsilon = 0.01$, начальной точке (-5.0, 6.0) методом сопряженных градиентов}
	        \vspace*{-1.2cm}
            \end{figure}
            
            \begin{figure}[H]
	        \centering
	        \includegraphics[width=0.70\textwidth]{Метод Флетчера-Ривса, eps 0.01, start = (-5.00, 6.00), Функция Розенброка с alpha = 1}%
	        \caption{Поиск минимума функции Розенброка с $\alpha$ = 1 при $\varepsilon = 0.01$, начальной точке (-5.0, 6.0) методом Флетчера --- Ривса}
	        \vspace*{-1.2cm}
            \end{figure}
            
            \begin{figure}[H]
	        \centering
	        \includegraphics[width=0.70\textwidth]{Метод Полака-Рибера, eps 0.01, start = (-5.00, 6.00), Функция Розенброка с alpha = 1}%
	        \caption{Поиск минимума функции Розенброка с $\alpha$ = 1 при $\varepsilon = 0.01$, начальной точке (-5.0, 6.0) методом Полака --- Рибера}
	        \vspace*{-1.2cm}
            \end{figure}
            
            \begin{figure}[H]
	        \centering
	        \includegraphics[width=0.70\textwidth]{Метод сопряженных градиентов, eps 1e-06, start = (3.000000, 20.000000), Функция Розенброка с alpha = 1}%
	        \caption{Поиск минимума функции Розенброка с $\alpha$ = 1 при $\varepsilon = 1e-06$, начальной точке (3.0, 20.0) методом сопряженных градиентов}
	        \vspace*{-1.2cm}
            \end{figure}
            
            \begin{figure}[H]
	        \centering
	        \includegraphics[width=0.70\textwidth]{Метод Флетчера-Ривса, eps 1e-06, start = (3.000000, 20.000000), Функция Розенброка с alpha = 1}%
	        \caption{Поиск минимума функции Розенброка с $\alpha$ = 1 при $\varepsilon = 1e-06$, начальной точке (3.0, 20.0) методом Флетчера --- Ривса}
	        \vspace*{-1.2cm}
            \end{figure}
            
            \begin{figure}[H]
	        \centering
	        \includegraphics[width=0.70\textwidth]{Метод Полака-Рибера, eps 1e-06, start = (3.000000, 20.000000), Функция Розенброка с alpha = 1}%
	        \caption{Поиск минимума функции Розенброка с $\alpha$ = 1 при $\varepsilon = 1e-06$, начальной точке (3.0, 20.0) методом Полака --- Рибера}
	        \vspace*{-1.2cm}
            \end{figure}
            
            \begin{figure}[H]
	        \centering
	        \includegraphics[width=0.70\textwidth]{Метод сопряженных градиентов, eps 1e-06, start = (-5.000000, 6.000000), Функция Розенброка с alpha = 1}%
	        \caption{Поиск минимума функции Розенброка с $\alpha$ = 1 при $\varepsilon = 1e-06$, начальной точке (-5.0, 6.0) методом сопряженных градиентов}
	        \vspace*{-1.2cm}
            \end{figure}
            
            \begin{figure}[H]
	        \centering
	        \includegraphics[width=0.70\textwidth]{Метод Флетчера-Ривса, eps 1e-06, start = (-5.000000, 6.000000), Функция Розенброка с alpha = 1}%
	        \caption{Поиск минимума функции Розенброка с $\alpha$ = 1 при $\varepsilon = 1e-06$, начальной точке (-5.0, 6.0) методом Флетчера --- Ривса}
	        \vspace*{-1.2cm}
            \end{figure}
            
            \begin{figure}[H]
	        \centering
	        \includegraphics[width=0.70\textwidth]{Метод Полака-Рибера, eps 1e-06, start = (-5.000000, 6.000000), Функция Розенброка с alpha = 1}%
	        \caption{Поиск минимума функции Розенброка с $\alpha$ = 1 при $\varepsilon = 1e-06$, начальной точке (-5.0, 6.0) методом Полака --- Рибера}
	        \vspace*{-1.2cm}
            \end{figure}
            \subsection{Функция Розенброка с $\alpha$ = 10}

\begin{table}[H]
        \centering
        \vspace*{-1.5em}
        \caption{Результаты работы алгоритмов\\для функции Розенброка с $\alpha$ = 10}
        \footnotesize
        \begin{tabular}{|c|c|c|c|c|}
        \hline
        & &\makecell{Метод\\сопряженных\\градиентов} &\makecell{Метод\\Флетчера --- Ривса} &\makecell{Метод\\Полака --- Рибера} \\
        \hline
	\multirow{12}{*}{\rotatebox[origin=c]{90}{$\varepsilon = 0.01$}}&\textbf{Начальная точка} &\multicolumn{3}{c|}{\textbf{(3.00, 20.00)}}\\
	\cline{2-5}
	&Точка минимума &(1.00, 1.00) &(0.99, 0.99) &(1.00, 1.00) \\ 
	\cline{2-5}
	&Минимум &0.00 &0.00 &0.00 \\ 
	\cline{2-5}
	&Кол-во итераций &26 &203 &29 \\ 
	\cline{2-5}
	&\makecell{Кол-во вызовов\\целевой функции} &479 &4641 &661 \\ 
	\cline{2-5}
	&\makecell{Кол-во вычислений\\градиента} &27 &204 &30 \\ 
	\cline{2-5}
	&\makecell{Кол-во вычислений\\матриц Гессе} &26 &0 &0 \\ 
	\cline{2-5}
\cline{2-5}&\textbf{Начальная точка} &\multicolumn{3}{c|}{\textbf{(-5.00, 6.00)}}\\
	\cline{2-5}
	&Точка минимума &(1.00, 1.00) &(1.00, 1.00) &(1.00, 1.00) \\ 
	\cline{2-5}
	&Минимум &0.00 &0.00 &0.00 \\ 
	\cline{2-5}
	&Кол-во итераций &29 &59 &16 \\ 
	\cline{2-5}
	&\makecell{Кол-во вызовов\\целевой функции} &515 &1256 &327 \\ 
	\cline{2-5}
	&\makecell{Кол-во вычислений\\градиента} &30 &60 &17 \\ 
	\cline{2-5}
	&\makecell{Кол-во вычислений\\матриц Гессе} &29 &0 &0 \\ 
	\cline{2-5}
	\hline
	\multirow{12}{*}{\rotatebox[origin=c]{90}{$\varepsilon = 1e-06$}}&\textbf{Начальная точка} &\multicolumn{3}{c|}{\textbf{(3.000000, 20.000000)}}\\
	\cline{2-5}
	&Точка минимума &(1.000000, 1.000000) &(1.000000, 1.000000) &(1.000000, 1.000000) \\ 
	\cline{2-5}
	&Минимум &0.000000 &0.000000 &0.000000 \\ 
	\cline{2-5}
	&Кол-во итераций &28 &99 &39 \\ 
	\cline{2-5}
	&\makecell{Кол-во вызовов\\целевой функции} &1102 &3883 &1427 \\ 
	\cline{2-5}
	&\makecell{Кол-во вычислений\\градиента} &29 &100 &40 \\ 
	\cline{2-5}
	&\makecell{Кол-во вычислений\\матриц Гессе} &28 &0 &0 \\ 
	\cline{2-5}
\cline{2-5}&\textbf{Начальная точка} &\multicolumn{3}{c|}{\textbf{(-5.000000, 6.000000)}}\\
	\cline{2-5}
	&Точка минимума &(1.000001, 1.000002) &(1.000000, 0.999999) &(1.000000, 1.000000) \\ 
	\cline{2-5}
	&Минимум &0.000000 &0.000000 &0.000000 \\ 
	\cline{2-5}
	&Кол-во итераций &80 &78 &19 \\ 
	\cline{2-5}
	&\makecell{Кол-во вызовов\\целевой функции} &2876 &2977 &694 \\ 
	\cline{2-5}
	&\makecell{Кол-во вычислений\\градиента} &81 &79 &20 \\ 
	\cline{2-5}
	&\makecell{Кол-во вычислений\\матриц Гессе} &80 &0 &0 \\ 
	\cline{2-5}
	\hline

\end{tabular}
\end{table}


            \begin{figure}[H]
	        \centering
	        \includegraphics[width=0.70\textwidth]{Метод сопряженных градиентов, eps 0.01, start = (3.00, 20.00), Функция Розенброка с alpha = 10}%
	        \caption{Поиск минимума функции Розенброка с $\alpha$ = 10 при $\varepsilon = 0.01$, начальной точке (3.0, 20.0) методом сопряженных градиентов}
	        \vspace*{-1.2cm}
            \end{figure}
            
            \begin{figure}[H]
	        \centering
	        \includegraphics[width=0.70\textwidth]{Метод Флетчера-Ривса, eps 0.01, start = (3.00, 20.00), Функция Розенброка с alpha = 10}%
	        \caption{Поиск минимума функции Розенброка с $\alpha$ = 10 при $\varepsilon = 0.01$, начальной точке (3.0, 20.0) методом Флетчера --- Ривса}
	        \vspace*{-1.2cm}
            \end{figure}
            
            \begin{figure}[H]
	        \centering
	        \includegraphics[width=0.70\textwidth]{Метод Полака-Рибера, eps 0.01, start = (3.00, 20.00), Функция Розенброка с alpha = 10}%
	        \caption{Поиск минимума функции Розенброка с $\alpha$ = 10 при $\varepsilon = 0.01$, начальной точке (3.0, 20.0) методом Полака --- Рибера}
	        \vspace*{-1.2cm}
            \end{figure}
            
            \begin{figure}[H]
	        \centering
	        \includegraphics[width=0.70\textwidth]{Метод сопряженных градиентов, eps 0.01, start = (-5.00, 6.00), Функция Розенброка с alpha = 10}%
	        \caption{Поиск минимума функции Розенброка с $\alpha$ = 10 при $\varepsilon = 0.01$, начальной точке (-5.0, 6.0) методом сопряженных градиентов}
	        \vspace*{-1.2cm}
            \end{figure}
            
            \begin{figure}[H]
	        \centering
	        \includegraphics[width=0.70\textwidth]{Метод Флетчера-Ривса, eps 0.01, start = (-5.00, 6.00), Функция Розенброка с alpha = 10}%
	        \caption{Поиск минимума функции Розенброка с $\alpha$ = 10 при $\varepsilon = 0.01$, начальной точке (-5.0, 6.0) методом Флетчера --- Ривса}
	        \vspace*{-1.2cm}
            \end{figure}
            
            \begin{figure}[H]
	        \centering
	        \includegraphics[width=0.70\textwidth]{Метод Полака-Рибера, eps 0.01, start = (-5.00, 6.00), Функция Розенброка с alpha = 10}%
	        \caption{Поиск минимума функции Розенброка с $\alpha$ = 10 при $\varepsilon = 0.01$, начальной точке (-5.0, 6.0) методом Полака --- Рибера}
	        \vspace*{-1.2cm}
            \end{figure}
            
            \begin{figure}[H]
	        \centering
	        \includegraphics[width=0.70\textwidth]{Метод сопряженных градиентов, eps 1e-06, start = (3.000000, 20.000000), Функция Розенброка с alpha = 10}%
	        \caption{Поиск минимума функции Розенброка с $\alpha$ = 10 при $\varepsilon = 1e-06$, начальной точке (3.0, 20.0) методом сопряженных градиентов}
	        \vspace*{-1.2cm}
            \end{figure}
            
            \begin{figure}[H]
	        \centering
	        \includegraphics[width=0.70\textwidth]{Метод Флетчера-Ривса, eps 1e-06, start = (3.000000, 20.000000), Функция Розенброка с alpha = 10}%
	        \caption{Поиск минимума функции Розенброка с $\alpha$ = 10 при $\varepsilon = 1e-06$, начальной точке (3.0, 20.0) методом Флетчера --- Ривса}
	        \vspace*{-1.2cm}
            \end{figure}
            
            \begin{figure}[H]
	        \centering
	        \includegraphics[width=0.70\textwidth]{Метод Полака-Рибера, eps 1e-06, start = (3.000000, 20.000000), Функция Розенброка с alpha = 10}%
	        \caption{Поиск минимума функции Розенброка с $\alpha$ = 10 при $\varepsilon = 1e-06$, начальной точке (3.0, 20.0) методом Полака --- Рибера}
	        \vspace*{-1.2cm}
            \end{figure}
            
            \begin{figure}[H]
	        \centering
	        \includegraphics[width=0.70\textwidth]{Метод сопряженных градиентов, eps 1e-06, start = (-5.000000, 6.000000), Функция Розенброка с alpha = 10}%
	        \caption{Поиск минимума функции Розенброка с $\alpha$ = 10 при $\varepsilon = 1e-06$, начальной точке (-5.0, 6.0) методом сопряженных градиентов}
	        \vspace*{-1.2cm}
            \end{figure}
            
            \begin{figure}[H]
	        \centering
	        \includegraphics[width=0.70\textwidth]{Метод Флетчера-Ривса, eps 1e-06, start = (-5.000000, 6.000000), Функция Розенброка с alpha = 10}%
	        \caption{Поиск минимума функции Розенброка с $\alpha$ = 10 при $\varepsilon = 1e-06$, начальной точке (-5.0, 6.0) методом Флетчера --- Ривса}
	        \vspace*{-1.2cm}
            \end{figure}
            
            \begin{figure}[H]
	        \centering
	        \includegraphics[width=0.70\textwidth]{Метод Полака-Рибера, eps 1e-06, start = (-5.000000, 6.000000), Функция Розенброка с alpha = 10}%
	        \caption{Поиск минимума функции Розенброка с $\alpha$ = 10 при $\varepsilon = 1e-06$, начальной точке (-5.0, 6.0) методом Полака --- Рибера}
	        \vspace*{-1.2cm}
            \end{figure}
            