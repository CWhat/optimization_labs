\subsubsection{Квадратичная функция}

\begin{table}[H]
        \centering
        \vspace*{-1.5em}
        \caption{Результаты работы алгоритмов\\для квадратичной функции}
        \footnotesize
        \begin{tabular}{|c|c|c|c|}
        \hline
        & &\makecell{Метод внутренних\\штрафных функций} &\makecell{Метод внешних\\штрафных функций} \\
        \hline
	\multirow{8}{*}{\rotatebox[origin=c]{90}{$\varepsilon = 0.01$}}&\textbf{Начальная точка} &\multicolumn{2}{c|}{\textbf{(2.00, 3.00)}}\\
	\cline{2-4}
	&Точка минимума &(2.24, -0.00) &(2.24, -0.00) \\ 
	\cline{2-4}
	&Минимум &-66.00 &-66.00 \\ 
	\cline{2-4}
	&Кол-во итераций &16 &2 \\ 
	\cline{2-4}
	&\makecell{Кол-во вызовов\\целевой функции} &1127 &174 \\ 
	\cline{2-4}
\cline{2-4}&\textbf{Начальная точка} &\multicolumn{2}{c|}{\textbf{(7.00, 1.00)}}\\
	\cline{2-4}
	&Точка минимума &(2.24, -0.00) &(2.24, -0.00) \\ 
	\cline{2-4}
	&Минимум &-66.00 &-66.00 \\ 
	\cline{2-4}
	&Кол-во итераций &16 &2 \\ 
	\cline{2-4}
	&\makecell{Кол-во вызовов\\целевой функции} &1137 &179 \\ 
	\cline{2-4}
	\hline
	\multirow{8}{*}{\rotatebox[origin=c]{90}{$\varepsilon = 1e-06$}}&\textbf{Начальная точка} &\multicolumn{2}{c|}{\textbf{(2.000000, 3.000000)}}\\
	\cline{2-4}
	&Точка минимума &(2.236068, -0.000000) &(2.236068, 0.000000) \\ 
	\cline{2-4}
	&Минимум &-66.000000 &-66.000000 \\ 
	\cline{2-4}
	&Кол-во итераций &29 &2 \\ 
	\cline{2-4}
	&\makecell{Кол-во вызовов\\целевой функции} &3752 &336 \\ 
	\cline{2-4}
\cline{2-4}&\textbf{Начальная точка} &\multicolumn{2}{c|}{\textbf{(7.000000, 1.000000)}}\\
	\cline{2-4}
	&Точка минимума &(2.236068, -0.000000) &(2.236068, -0.000000) \\ 
	\cline{2-4}
	&Минимум &-66.000000 &-66.000000 \\ 
	\cline{2-4}
	&Кол-во итераций &29 &2 \\ 
	\cline{2-4}
	&\makecell{Кол-во вызовов\\целевой функции} &3722 &341 \\ 
	\cline{2-4}
	\hline

\end{tabular}
\end{table}


            \begin{figure}[H]
	        \centering
	        \includegraphics[width=0.85\textwidth]{Метод внутренних штрафных функций, eps 0.01, start = (2.00, 3.00), Квадратичная функция, testa}%
	        \caption{Поиск минимума квадратичной функции при $\varepsilon = 0.01$, начальной точке (2.0, 3.0) методом внутренних штрафных функций}
	        \vspace*{-1.2cm}
            \end{figure}
            
            \begin{figure}[H]
	        \centering
	        \includegraphics[width=0.85\textwidth]{Метод внешних штрафных функций, eps 0.01, start = (2.00, 3.00), Квадратичная функция, testa}%
	        \caption{Поиск минимума квадратичной функции при $\varepsilon = 0.01$, начальной точке (2.0, 3.0) методом внешних штрафных функций}
	        \vspace*{-1.2cm}
            \end{figure}
            
            \begin{figure}[H]
	        \centering
	        \includegraphics[width=0.85\textwidth]{Метод внутренних штрафных функций, eps 0.01, start = (7.00, 1.00), Квадратичная функция, testa}%
	        \caption{Поиск минимума квадратичной функции при $\varepsilon = 0.01$, начальной точке (7.0, 1.0) методом внутренних штрафных функций}
	        \vspace*{-1.2cm}
            \end{figure}
            
            \begin{figure}[H]
	        \centering
	        \includegraphics[width=0.85\textwidth]{Метод внешних штрафных функций, eps 0.01, start = (7.00, 1.00), Квадратичная функция, testa}%
	        \caption{Поиск минимума квадратичной функции при $\varepsilon = 0.01$, начальной точке (7.0, 1.0) методом внешних штрафных функций}
	        \vspace*{-1.2cm}
            \end{figure}
            
            \begin{figure}[H]
	        \centering
	        \includegraphics[width=0.85\textwidth]{Метод внутренних штрафных функций, eps 1e-06, start = (2.000000, 3.000000), Квадратичная функция, testa}%
	        \caption{Поиск минимума квадратичной функции при $\varepsilon = 1e-06$, начальной точке (2.0, 3.0) методом внутренних штрафных функций}
	        \vspace*{-1.2cm}
            \end{figure}
            
            \begin{figure}[H]
	        \centering
	        \includegraphics[width=0.85\textwidth]{Метод внешних штрафных функций, eps 1e-06, start = (2.000000, 3.000000), Квадратичная функция, testa}%
	        \caption{Поиск минимума квадратичной функции при $\varepsilon = 1e-06$, начальной точке (2.0, 3.0) методом внешних штрафных функций}
	        \vspace*{-1.2cm}
            \end{figure}
            
            \begin{figure}[H]
	        \centering
	        \includegraphics[width=0.85\textwidth]{Метод внутренних штрафных функций, eps 1e-06, start = (7.000000, 1.000000), Квадратичная функция, testa}%
	        \caption{Поиск минимума квадратичной функции при $\varepsilon = 1e-06$, начальной точке (7.0, 1.0) методом внутренних штрафных функций}
	        \vspace*{-1.2cm}
            \end{figure}
            
            \begin{figure}[H]
	        \centering
	        \includegraphics[width=0.85\textwidth]{Метод внешних штрафных функций, eps 1e-06, start = (7.000000, 1.000000), Квадратичная функция, testa}%
	        \caption{Поиск минимума квадратичной функции при $\varepsilon = 1e-06$, начальной точке (7.0, 1.0) методом внешних штрафных функций}
	        \vspace*{-1.2cm}
            \end{figure}
            \subsubsection{Функция Розенброка с $\alpha$ = 1}

\begin{table}[H]
        \centering
        \vspace*{-1.5em}
        \caption{Результаты работы алгоритмов\\для функции Розенброка с $\alpha$ = 1}
        \footnotesize
        \begin{tabular}{|c|c|c|c|}
        \hline
        & &\makecell{Метод внутренних\\штрафных функций} &\makecell{Метод внешних\\штрафных функций} \\
        \hline
	\multirow{8}{*}{\rotatebox[origin=c]{90}{$\varepsilon = 0.01$}}&\textbf{Начальная точка} &\multicolumn{2}{c|}{\textbf{(2.00, 3.00)}}\\
	\cline{2-4}
	&Точка минимума &(1.00, 1.00) &(1.00, 1.00) \\ 
	\cline{2-4}
	&Минимум &0.00 &0.00 \\ 
	\cline{2-4}
	&Кол-во итераций &9 &2 \\ 
	\cline{2-4}
	&\makecell{Кол-во вызовов\\целевой функции} &1583 &364 \\ 
	\cline{2-4}
\cline{2-4}&\textbf{Начальная точка} &\multicolumn{2}{c|}{\textbf{(7.00, 1.00)}}\\
	\cline{2-4}
	&Точка минимума &(1.00, 1.00) &(1.00, 1.00) \\ 
	\cline{2-4}
	&Минимум &0.00 &0.00 \\ 
	\cline{2-4}
	&Кол-во итераций &9 &2 \\ 
	\cline{2-4}
	&\makecell{Кол-во вызовов\\целевой функции} &1643 &474 \\ 
	\cline{2-4}
	\hline
	\multirow{8}{*}{\rotatebox[origin=c]{90}{$\varepsilon = 1e-06$}}&\textbf{Начальная точка} &\multicolumn{2}{c|}{\textbf{(2.000000, 3.000000)}}\\
	\cline{2-4}
	&Точка минимума &(1.000000, 1.000000) &(1.000000, 1.000000) \\ 
	\cline{2-4}
	&Минимум &0.000000 &0.000000 \\ 
	\cline{2-4}
	&Кол-во итераций &15 &2 \\ 
	\cline{2-4}
	&\makecell{Кол-во вызовов\\целевой функции} &4590 &641 \\ 
	\cline{2-4}
\cline{2-4}&\textbf{Начальная точка} &\multicolumn{2}{c|}{\textbf{(7.000000, 1.000000)}}\\
	\cline{2-4}
	&Точка минимума &(1.000000, 1.000000) &(1.000000, 1.000000) \\ 
	\cline{2-4}
	&Минимум &0.000000 &0.000000 \\ 
	\cline{2-4}
	&Кол-во итераций &15 &2 \\ 
	\cline{2-4}
	&\makecell{Кол-во вызовов\\целевой функции} &4630 &706 \\ 
	\cline{2-4}
	\hline

\end{tabular}
\end{table}


            \begin{figure}[H]
	        \centering
	        \includegraphics[width=0.85\textwidth]{Метод внутренних штрафных функций, eps 0.01, start = (2.00, 3.00), Функция Розенброка с alpha = 1, testa}%
	        \caption{Поиск минимума функции Розенброка с $\alpha$ = 1 при $\varepsilon = 0.01$, начальной точке (2.0, 3.0) методом внутренних штрафных функций}
	        \vspace*{-1.2cm}
            \end{figure}
            
            \begin{figure}[H]
	        \centering
	        \includegraphics[width=0.85\textwidth]{Метод внешних штрафных функций, eps 0.01, start = (2.00, 3.00), Функция Розенброка с alpha = 1, testa}%
	        \caption{Поиск минимума функции Розенброка с $\alpha$ = 1 при $\varepsilon = 0.01$, начальной точке (2.0, 3.0) методом внешних штрафных функций}
	        \vspace*{-1.2cm}
            \end{figure}
            
            \begin{figure}[H]
	        \centering
	        \includegraphics[width=0.85\textwidth]{Метод внутренних штрафных функций, eps 0.01, start = (7.00, 1.00), Функция Розенброка с alpha = 1, testa}%
	        \caption{Поиск минимума функции Розенброка с $\alpha$ = 1 при $\varepsilon = 0.01$, начальной точке (7.0, 1.0) методом внутренних штрафных функций}
	        \vspace*{-1.2cm}
            \end{figure}
            
            \begin{figure}[H]
	        \centering
	        \includegraphics[width=0.85\textwidth]{Метод внешних штрафных функций, eps 0.01, start = (7.00, 1.00), Функция Розенброка с alpha = 1, testa}%
	        \caption{Поиск минимума функции Розенброка с $\alpha$ = 1 при $\varepsilon = 0.01$, начальной точке (7.0, 1.0) методом внешних штрафных функций}
	        \vspace*{-1.2cm}
            \end{figure}
            
            \begin{figure}[H]
	        \centering
	        \includegraphics[width=0.85\textwidth]{Метод внутренних штрафных функций, eps 1e-06, start = (2.000000, 3.000000), Функция Розенброка с alpha = 1, testa}%
	        \caption{Поиск минимума функции Розенброка с $\alpha$ = 1 при $\varepsilon = 1e-06$, начальной точке (2.0, 3.0) методом внутренних штрафных функций}
	        \vspace*{-1.2cm}
            \end{figure}
            
            \begin{figure}[H]
	        \centering
	        \includegraphics[width=0.85\textwidth]{Метод внешних штрафных функций, eps 1e-06, start = (2.000000, 3.000000), Функция Розенброка с alpha = 1, testa}%
	        \caption{Поиск минимума функции Розенброка с $\alpha$ = 1 при $\varepsilon = 1e-06$, начальной точке (2.0, 3.0) методом внешних штрафных функций}
	        \vspace*{-1.2cm}
            \end{figure}
            
            \begin{figure}[H]
	        \centering
	        \includegraphics[width=0.85\textwidth]{Метод внутренних штрафных функций, eps 1e-06, start = (7.000000, 1.000000), Функция Розенброка с alpha = 1, testa}%
	        \caption{Поиск минимума функции Розенброка с $\alpha$ = 1 при $\varepsilon = 1e-06$, начальной точке (7.0, 1.0) методом внутренних штрафных функций}
	        \vspace*{-1.2cm}
            \end{figure}
            
            \begin{figure}[H]
	        \centering
	        \includegraphics[width=0.85\textwidth]{Метод внешних штрафных функций, eps 1e-06, start = (7.000000, 1.000000), Функция Розенброка с alpha = 1, testa}%
	        \caption{Поиск минимума функции Розенброка с $\alpha$ = 1 при $\varepsilon = 1e-06$, начальной точке (7.0, 1.0) методом внешних штрафных функций}
	        \vspace*{-1.2cm}
            \end{figure}
            \subsubsection{Функция Розенброка с $\alpha$ = 10}

\begin{table}[H]
        \centering
        \vspace*{-1.5em}
        \caption{Результаты работы алгоритмов\\для функции Розенброка с $\alpha$ = 10}
        \footnotesize
        \begin{tabular}{|c|c|c|c|}
        \hline
        & &\makecell{Метод внутренних\\штрафных функций} &\makecell{Метод внешних\\штрафных функций} \\
        \hline
	\multirow{8}{*}{\rotatebox[origin=c]{90}{$\varepsilon = 0.01$}}&\textbf{Начальная точка} &\multicolumn{2}{c|}{\textbf{(2.00, 3.00)}}\\
	\cline{2-4}
	&Точка минимума &(1.00, 1.00) &(1.00, 1.00) \\ 
	\cline{2-4}
	&Минимум &0.00 &0.00 \\ 
	\cline{2-4}
	&Кол-во итераций &9 &2 \\ 
	\cline{2-4}
	&\makecell{Кол-во вызовов\\целевой функции} &8028 &2259 \\ 
	\cline{2-4}
\cline{2-4}&\textbf{Начальная точка} &\multicolumn{2}{c|}{\textbf{(7.00, 1.00)}}\\
	\cline{2-4}
	&Точка минимума &(1.00, 1.01) &(1.00, 1.01) \\ 
	\cline{2-4}
	&Минимум &0.00 &0.00 \\ 
	\cline{2-4}
	&Кол-во итераций &9 &2 \\ 
	\cline{2-4}
	&\makecell{Кол-во вызовов\\целевой функции} &7963 &2734 \\ 
	\cline{2-4}
	\hline
	\multirow{8}{*}{\rotatebox[origin=c]{90}{$\varepsilon = 1e-06$}}&\textbf{Начальная точка} &\multicolumn{2}{c|}{\textbf{(2.000000, 3.000000)}}\\
	\cline{2-4}
	&Точка минимума &(1.000000, 1.000001) &(1.000000, 1.000001) \\ 
	\cline{2-4}
	&Минимум &0.000000 &0.000000 \\ 
	\cline{2-4}
	&Кол-во итераций &15 &2 \\ 
	\cline{2-4}
	&\makecell{Кол-во вызовов\\целевой функции} &29490 &3696 \\ 
	\cline{2-4}
\cline{2-4}&\textbf{Начальная точка} &\multicolumn{2}{c|}{\textbf{(7.000000, 1.000000)}}\\
	\cline{2-4}
	&Точка минимума &(1.000000, 1.000001) &(1.000000, 1.000000) \\ 
	\cline{2-4}
	&Минимум &0.000000 &0.000000 \\ 
	\cline{2-4}
	&Кол-во итераций &15 &2 \\ 
	\cline{2-4}
	&\makecell{Кол-во вызовов\\целевой функции} &29160 &4276 \\ 
	\cline{2-4}
	\hline

\end{tabular}
\end{table}


            \begin{figure}[H]
	        \centering
	        \includegraphics[width=0.85\textwidth]{Метод внутренних штрафных функций, eps 0.01, start = (2.00, 3.00), Функция Розенброка с alpha = 10, testa}%
	        \caption{Поиск минимума функции Розенброка с $\alpha$ = 10 при $\varepsilon = 0.01$, начальной точке (2.0, 3.0) методом внутренних штрафных функций}
	        \vspace*{-1.2cm}
            \end{figure}
            
            \begin{figure}[H]
	        \centering
	        \includegraphics[width=0.85\textwidth]{Метод внешних штрафных функций, eps 0.01, start = (2.00, 3.00), Функция Розенброка с alpha = 10, testa}%
	        \caption{Поиск минимума функции Розенброка с $\alpha$ = 10 при $\varepsilon = 0.01$, начальной точке (2.0, 3.0) методом внешних штрафных функций}
	        \vspace*{-1.2cm}
            \end{figure}
            
            \begin{figure}[H]
	        \centering
	        \includegraphics[width=0.85\textwidth]{Метод внутренних штрафных функций, eps 0.01, start = (7.00, 1.00), Функция Розенброка с alpha = 10, testa}%
	        \caption{Поиск минимума функции Розенброка с $\alpha$ = 10 при $\varepsilon = 0.01$, начальной точке (7.0, 1.0) методом внутренних штрафных функций}
	        \vspace*{-1.2cm}
            \end{figure}
            
            \begin{figure}[H]
	        \centering
	        \includegraphics[width=0.85\textwidth]{Метод внешних штрафных функций, eps 0.01, start = (7.00, 1.00), Функция Розенброка с alpha = 10, testa}%
	        \caption{Поиск минимума функции Розенброка с $\alpha$ = 10 при $\varepsilon = 0.01$, начальной точке (7.0, 1.0) методом внешних штрафных функций}
	        \vspace*{-1.2cm}
            \end{figure}
            
            \begin{figure}[H]
	        \centering
	        \includegraphics[width=0.85\textwidth]{Метод внутренних штрафных функций, eps 1e-06, start = (2.000000, 3.000000), Функция Розенброка с alpha = 10, testa}%
	        \caption{Поиск минимума функции Розенброка с $\alpha$ = 10 при $\varepsilon = 1e-06$, начальной точке (2.0, 3.0) методом внутренних штрафных функций}
	        \vspace*{-1.2cm}
            \end{figure}
            
            \begin{figure}[H]
	        \centering
	        \includegraphics[width=0.85\textwidth]{Метод внешних штрафных функций, eps 1e-06, start = (2.000000, 3.000000), Функция Розенброка с alpha = 10, testa}%
	        \caption{Поиск минимума функции Розенброка с $\alpha$ = 10 при $\varepsilon = 1e-06$, начальной точке (2.0, 3.0) методом внешних штрафных функций}
	        \vspace*{-1.2cm}
            \end{figure}
            
            \begin{figure}[H]
	        \centering
	        \includegraphics[width=0.85\textwidth]{Метод внутренних штрафных функций, eps 1e-06, start = (7.000000, 1.000000), Функция Розенброка с alpha = 10, testa}%
	        \caption{Поиск минимума функции Розенброка с $\alpha$ = 10 при $\varepsilon = 1e-06$, начальной точке (7.0, 1.0) методом внутренних штрафных функций}
	        \vspace*{-1.2cm}
            \end{figure}
            
            \begin{figure}[H]
	        \centering
	        \includegraphics[width=0.85\textwidth]{Метод внешних штрафных функций, eps 1e-06, start = (7.000000, 1.000000), Функция Розенброка с alpha = 10, testa}%
	        \caption{Поиск минимума функции Розенброка с $\alpha$ = 10 при $\varepsilon = 1e-06$, начальной точке (7.0, 1.0) методом внешних штрафных функций}
	        \vspace*{-1.2cm}
            \end{figure}
            